\documentclass{article}
%==============================================================================%
%	                          Packages                                     %
%==============================================================================%
% Packages
\usepackage[utf8]{inputenc}
\usepackage{graphicx}
\usepackage{amsmath}
\usepackage{amssymb}
\usepackage{braket}
\usepackage[margin=0.7in]{geometry}
\usepackage[version=4]{mhchem}
%==============================================================================%
%                           User-Defined Commands                              %
%==============================================================================%
% User-Defined Commands
\newcommand{\be}{\begin{equation}}
\newcommand{\ee}{\end{equation}}
\newcommand{\benum}{\begin{enumerate}}
\newcommand{\eenum}{\end{enumerate}}
\newcommand{\pd}{\partial}
\newcommand{\dg}{\dagger}
%==============================================================================%
%                             Title Information                                %
%==============================================================================%
\title{Chem237: Lecture 5}
\date{4/11/18}
\author{Shane Flynn,Moises Romero}
%==============================================================================%
%	Everyone Please Make Comments if Something Needs to be Reviewed        %
%                           Or just fix it yourself!                           %
%==============================================================================%
\begin{document}
\maketitle
\section*{Approximate Bounds}
%==============================================================================%
%                           This Lecture begins with approximate integral methods , %
% add in later
%==============================================================================%
\section*{Fourier Transform}
A Fourier transform is a generalization of the complex Fourier series as the limit L $\to \infty$ defined as :
\be
g(y) = \int_{\- infty}^\infty f(x) e^{ixy} dx
\ee
And its inverse Fourier transform is defined by :
\be
f(x) = \int_{-\infty}^{\infty} g(y) e^{-ixy} dy
\ee
Before this is discussed more thoroughly we will talk about Fourier Series.
\subsection*{Fourier Series}
A Fourier series is a method to write a periodic function in terms of Sines Cosines. A fourier series is written in the form of :
\be
f(\theta) = \frac{A_o}{2} + \sum_{n=1}^\infty (A_n Cos(n\theta) + B_n Sin(n \theta) )
\ee
The coefficients $A_n$ and $B_n$ can be found by multiplying the function by multiply each by Cos(n$\theta$) and Sin(n$\theta$) :
\be
A_n = \frac{1}{\pi} \int_0^{2\pi} f(\theta) Cos(n\theta) d\theta
\ee
\be
B_n = \frac{1}{\pi} \int_0^{2\pi} f(\theta) Sin(n\theta) d\theta
\ee
The $A_0$ term can be found by plugging in n=0 and getting :
\be
A_0 = \frac{2}{L} \int_0^L f(x) dx
\ee
Where L is the period.
\end{document}

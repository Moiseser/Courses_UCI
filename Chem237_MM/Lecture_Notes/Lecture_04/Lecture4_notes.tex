\documentclass{article}
%==============================================================================%
%	                          Packages                                     %
%==============================================================================%
% Packages
\usepackage[utf8]{inputenc}
\usepackage{graphicx}
\usepackage{amsmath}
\usepackage{amssymb}
\usepackage{braket}
\usepackage[margin=0.7in]{geometry}
\usepackage[version=4]{mhchem}
%==============================================================================%
%                           User-Defined Commands                              %
%==============================================================================%
% User-Defined Commands
\newcommand{\be}{\begin{equation}}
\newcommand{\ee}{\end{equation}}
\newcommand{\benum}{\begin{enumerate}}
\newcommand{\eenum}{\end{enumerate}}
\newcommand{\pd}{\partial}
\newcommand{\dg}{\dagger}
\newcommand{\sums}{\sum_{n=0}^\infty}
%==============================================================================%
%                             Title Information                                %
%==============================================================================%
\title{Chem237: Lecture 4}
\date{4/10/18}
\author{Shane Flynn}
%==============================================================================%
%	Everyone Please Make Comments if Something Needs to be Reviewed        %
%                           Or just fix it yourself!                           %
%==============================================================================%
\begin{document}
\maketitle
\section*{More Series}
The alternating series has some interesting convergence properties, it takes the form
\be
\sums \frac{(-1)^{n+1}}{n} = \ln (1+1) = 2
\ee
If we start to look at some of the explicit values of the series we see 
\be
1 - \frac{1}{2} + \frac{1}{3} - \frac{1}{4} + \cdots
\ee
With general numbers we know that addition and subtraction commute i.e. A + B = B + A, so we would expect that if we re-arrange the order in which we take our infinite summation we would still produce teh same result at the end. 
So consider a fancy summation scheme for the alternating series where we group terms
\be
\left(1 + \frac{1}{3} + \frac{1}{5}\right) - \frac{1}{2} + \left(\frac{1}{7} + \frac{1}{9} + \frac{1}{11} + \frac{1}{13} + \frac{1}{15}\right) - \frac{1}{4} \cdots
\ee
So we bnreak the summation into positive and negative terms, defining a set of partial summations over the positive and then negative terms. 
If you numerically evaluate the summations in this way you will find that the summation appears to approach 1.5 as n goes to $\infty$.
In fact it turns out that you can re-arrange the terms in any sort of pattern, and find the series converges to any value you want. 
This is a consequence of the series not being absolutely convergent!
The commutation of terms applies to finite numbers, it is not true for an infinite series, meaning A + B $\neq$ B + A over an infinite interval, it is only true for finite numbers. 
If the series is absolutely convergent however, than the series converges to the same value no matter what, and changing the order of the summation will not effect the result. 

\subsection*{Power Series}
The purpose of the power series is to exapnd a function in terms of x. 
\be
\sums a_x x^n = f(x) 
\ee
If we have a Taylor series of a known function x than things are simple for example
%==============================================================================%
%                             Put in derivation for the Taylor expansions.     %
%==============================================================================%
\subsubsection*{Geometric}
\be
\sums x^n = \frac{1}{1-x}
\ee

\subsubsection*{Expotential}
\be
\sums \frac{x^n}{n!} = e^x
\ee

\subsubsection*{Sine}
\be
\sums \frac{x^{2k+1}(-1)^{k+1}}{(2k+1)!} = \sin(x)
\ee

\subsubsection*{Cosine}
\be
\sums \frac{x^{2k}(-1)^k}{(2k)!} = \cos(x)
\ee

\subsubsection*{Logarithm}
\be
\sums (-1)^{n+1} \frac{x^n}{n} = \ln(1+x)
\ee

If we are given a series we do not recognize, we can try applying a transformation to make it into a recognized form, such as taking a derivitive or an integral of the original series. 
\be
f(x) = \sum_{n=1}^\infty a_nx^n
\ee
We can write a new series, using the derivative.
\be
f'(x) = \sum_{n=1}^\infty na_nx^{n-1}
\ee
Now consider an integral
\be
\int_1^x dx f(x) = \sum_{n=1}^\infty \frac{a_n}{n+1}x^{n+1}
\ee

Consider for example
\be
\sum_{n=1}^\infty nx^n
\ee
This is similar to a geometric series in form. 
\be
\begin{split}
	\sums x^n = \frac{1}{1-x} &\Rightarrow \frac{d}{dx} \left( \sums x^n\right) = \frac{d}{dx} \left( \frac{1}{1-x} \right)\\
	\sums nx^{n-1} &= \frac{1}{(1-x)^2} \\
\end{split}
\ee
To make this lase line look like our example we just need to multiple by x and cahnge the summation bounds
\be
\sum_{n=1}^\infty nx^n = \frac{x}{(1-x)^2} 
\ee
%==============================================================================%
%             Vlad had a minus sign in his answer not sure which is correct 
%==============================================================================%
As another example this time consider
\be
\sum_{n=1}^\infty \frac{x^n}{n}
\ee
This expression again looks similar to a geometric series.
Consider what happens if we take an integral of a geometric series
\be
\sums x^n = \frac{1}{1-x} \Rightarrow \int \text{ } dx \frac{1}{1-x} = 
\ee

If we divide by x we reproduce the problem.
\be
\begin{split}
	\sums x^n &= \frac{1}{1-x} \Rightarrow \int \text{ } dx \frac{1}{1-x} \Rightarrow\\
	\int \text{ } dx \left(\frac{1}{1-x}\right)\left(\frac{1}{x}\right) &= \int \text{ } dx \sums x^n\left(\frac{1}{x}\right) = \sum_{n=1}^\infty \frac{x^{n+1}}{n} \left(\frac{1}{x}\right) \\
	\sum_{n=1}^\infty \frac{x^n}{n} &= \int \text{ } dx \left(\frac{1}{1-x}\right) \frac{1}{x} \Rightarrow \ln(1-x)
\end{split}
\ee

Another example, we can start using power series, consider a function of x.
\be
S(x) = \sum_{n=1}^\infty \frac{x^n}{n^2}
\ee
Where S(1) would be evaluated as
\be
S(1) = \sum_{n=1}^\infty \text{ }\frac{1}{n^2}
\ee

The solution of S(x) is related to the previous problem, if we take a derivative twice we will cancel terms. 
\be
\begin{split}
	S(x) &= \sum_{n=1}^\infty \frac{x^n}{n^2}
	S'(x) &= \sum_{n=1}^\infty \frac{x^{n-1}}{n}
\end{split}
\ee





































\end{document}

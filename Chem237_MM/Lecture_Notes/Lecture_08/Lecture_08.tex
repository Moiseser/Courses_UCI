\documentclass{article}
%==============================================================================%
%	                          Packages                                     %
%==============================================================================%
% Packages
\usepackage[utf8]{inputenc}
\usepackage{graphicx}
\usepackage{amsmath}
\usepackage{amssymb}
\usepackage{braket}
\usepackage[margin=0.7in]{geometry}
\usepackage[version=4]{mhchem}
%==============================================================================%
%                           User-Defined Commands                              %
%==============================================================================%
% User-Defined Commands
\newcommand{\be}{\begin{equation}}
\newcommand{\ee}{\end{equation}}
\newcommand{\benum}{\begin{enumerate}}
\newcommand{\eenum}{\end{enumerate}}
\newcommand{\pd}{\partial}
\newcommand{\dg}{\dagger}
\newcommand{\sumzero}{\sum_{n=0}^\infty}
\newcommand{\sumone}{\sum_{n=1}^\infty}
%==============================================================================%
%                             Title Information                                %
%==============================================================================%
\title{Chem237: Lecture 8}
\date{4/10/18}
\author{Alan Robledo}
%==============================================================================%
%	Everyone Please Make Comments if Something Needs to be Reviewed        %
%                           Or just fix it yourself!                           %
%==============================================================================%
\begin{document}
\maketitle

\subsection*{Approximate Methods}
Ever since Lecture 4, we discussed different ways to evaluate integrals of interest to give us exact answers.
It is often times enough for us to know an approximation to the value of te integral and we will discuss a few ways to do that by looking at some known special functions.
\subsubsection*{Error Function}
At first glance, you'd think that the error function looks like the integral of a gaussian
\be
  \text{erf(x)} = \frac{2}{\sqrt{\pi}} \int_0^x e^{-t^2} dt = \frac{1}{\sqrt{\pi}} \int_{-x}^x e^{-t^2} dt
\ee
and you'd be correct. To be specific, the error function is the integral of a normalized Gaussian with the normalization part coming from the $\frac{1}{\sqrt{\pi}}$.
The error function comes up in probability theory as this function is used to show the probability of something lying between -x and x.
Finding a good approximation of the integral can be done by taylor expanding the integrand about a center that we will take to be zero, and integrating term by term.
\be
  \begin{split}
    \text{erf(x)} &= \frac{2}{\sqrt{\pi}} \int_0^x e^{-t^2} dt \\
    &= \frac{2}{\sqrt{\pi}} \int_0^x (1 - t^2 + \frac{t^4}{2!} - \frac{t^6}{3!} + \hdots) dt \\
    &= \frac{2}{\sqrt{\pi}} (x - \frac{x^3}{3} + \frac{x^5}{5 \cdot 2!} - \frac{x^7}{7 \cdot 3!} + \hdots)
  \end{split}
\ee
By the look of it, we can say that the series converges for any x because the denominator tends to infinity much faster than the numerator ever could.
However, the convergence of the series is small for large x.
As a matter of fact, for large x, as x $\rightarrow \infty$, erf(x) $\rightarrow$ 1.
So suppose we wanted to know what erf(x) is for large x.
Using the fact that erf(x) $\rightarrow$ 1 as x $\rightarrow \infty$, we could write
\be
1 = \frac{2}{\sqrt{\pi}} \int_0^{\infty} e^{-t^2} dt
\ee
and if we break up the integrals
\be
1 = \frac{2}{\sqrt{\pi}} \int_0^x e^{-t^2} dt + \frac{2}{\sqrt{\pi}} \int_x^{\infty} e^{-t^2} dt
\ee
and rearrange terms, we get
\be \label{eq:erfc}
1 - \frac{2}{\sqrt{\pi}} \int_0^x e^{-t^2} dt = 1 - \text{erf(x)} = \frac{2}{\sqrt{\pi}} \int_x^{\infty} e^{-t^2} dt = \text{erfc(x)}
\ee
where we have an integral from large x to infinity, which happens to be called the complementary error function or erfc(x).
We can try to find an expression for the complementary error function by considering using integration by parts.
Now, you may look at the integrand and think to yourself that there are no two functions that you can use for integration by parts.
But a neat little trick that is often used is to consider multiplying the integrand by the one function.
\be
  \int_x^{\infty} e^{-t^2} dt = \int_x^{\infty} e^{-t^2} \cdot 1 dt = \int_x^{\infty} e^{-t^2} \cdot \frac{-2t}{-2t} dt
\ee
and if we consider,
\be
  \begin{split}
    U &= - \frac{1}{2t} \qquad dV = - 2t e^{-t^2} dt \\
    dU &= \frac{1}{2t^2} dt \qquad V = e^{-t^2}
  \end{split}
\ee
our integral becomes
\be
  \begin{split}
    \int_x^{\infty} e^{-t^2} dt &= \frac{e^{-t^2}}{2t} \Big|_x^{\infty} - \int_x^{\infty} \frac{e^{-t^2}}{2t^2} dt \\
    &= \frac{e^{-x^2}}{2x} - \int_x^{\infty} \frac{e^{-t^2}}{2t^2} dt
  \end{split}
\ee
We can integrate by parts again by using the same trick and multiplying the integrand by the one function
\be
  \int_x^{\infty} e^{-t^2} dt = \frac{e^{-x^2}}{2x} - \int_x^{\infty} \frac{e^{-t^2}}{2t^2} \cdot \frac{-2t}{-2t} dt
\ee
and considering
\be
  \begin{split}
    U &= - \frac{1}{4t^3} \qquad dV = - 2t e^{-t^2} dt \\
    dU &= \frac{3}{4t^4} dt \qquad V = e^{-t^2}
  \end{split}
\ee
to get
\be
  \begin{split}
    \int_x^{\infty} e^{-t^2} dt &= \frac{e^{-x^2}}{2x} - \int_x^{\infty} \frac{e^{-t^2}}{2t^2} \\
    &= \frac{e^{-x^2}}{2x} + \frac{e^{-t^2}}{4t^3} \Big|_x^{\infty} + \int_x^{\infty} \frac{3e^{-t^2}}{4t^4} dt \\
    &= \frac{e^{-x^2}}{2x} - \frac{e^{-x^2}}{4x^3} + \int_x^{\infty} \frac{3e^{-t^2}}{4t^4} dt
  \end{split}
\ee
You can see that we can use this trick each time we integrate by parts and if we integrated an infinite number of times, we'd have an exact expression.
If we go back to the equation \ref{eq:erfc} and solve for erf(x), we can see that the result after n integrations by parts of the complementary error function gives
\be
  \begin{split}
    \text{erf(x)} = 1 - \frac{2}{\sqrt{\pi}} \int_x^{\infty} e^{-t^2} dt = 1 - e^{-t^2} \Big[ \frac{1}{2x} - \frac{1}{2^2 x^3} & + \frac{1 \cdot 3}{2^3 x^5} - \frac{1 \cdot 3 \cdot 5}{2^4 x^7} + \hdots + (-1)^{n-1} \frac{1 \cdot 3 \cdot 5 \cdots (2n-3)}{2^n x^{2n-1}} \Big] \\
    & + (-1)^n \frac{1 \cdot 3 \cdot 5 \cdots (2n-1)}{2^n} \frac{2}{\sqrt{\pi}} \int_x^{\infty} \frac{e^{-t^2}}{t^{2n}} dt
  \end{split}
\ee
where the last term (the term after the brackets, including the integral) is the "remainder" that makes the expression exact.

\end{document}

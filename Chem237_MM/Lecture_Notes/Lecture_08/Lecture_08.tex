\documentclass{article}
%==============================================================================%
%	                          Packages                                     %
%==============================================================================%
% Packages
\usepackage[utf8]{inputenc}
\usepackage{graphicx}
\usepackage{amsmath}
\usepackage{amssymb}
\usepackage{braket}
\usepackage[margin=0.7in]{geometry}
\usepackage[version=4]{mhchem}
%==============================================================================%
%                           User-Defined Commands                              %
%==============================================================================%
% User-Defined Commands
\newcommand{\be}{\begin{equation}}
\newcommand{\ee}{\end{equation}}
\newcommand{\benum}{\begin{enumerate}}
\newcommand{\eenum}{\end{enumerate}}
\newcommand{\pd}{\partial}
\newcommand{\dg}{\dagger}
\newcommand{\sumzero}{\sum_{n=0}^\infty}
\newcommand{\sumone}{\sum_{n=1}^\infty}
%==============================================================================%
%                             Title Information                                %
%==============================================================================%
\title{Chem237: Lecture 8}
\date{4/10/18}
\author{Alan Robledo}
%==============================================================================%
%	Everyone Please Make Comments if Something Needs to be Reviewed        %
%                           Or just fix it yourself!                           %
%==============================================================================%
\begin{document}
\maketitle

\subsection*{Approximate Methods}
So far, we have discussed different ways to evaluate integrals of interest to obtain exact answers.
It is often times enough for us to know an approximation to the value of an integral and we will discuss a few methods that go about this by looking at some special functions as examples.
\subsubsection*{Asymptotic Series}
At first glance, you'd think that the error function is just the integral of a gaussian
\be
  \text{erf(x)} = \frac{2}{\sqrt{\pi}} \int_0^x e^{-t^2} dt = \frac{1}{\sqrt{\pi}} \int_{-x}^x e^{-t^2} dt
\ee
and you'd be correct. To be specific, the error function is the integral of a normalized Gaussian with the normalization part coming from the $\frac{1}{\sqrt{\pi}}$.
The error function comes up in probability theory as this function is used to show the probability of something lying between -x and x.
Finding a good approximation of the integral can be done by taylor expanding the integrand about a center that we will take to be zero, and integrating term by term.
\be
  \begin{split}
    \text{erf(x)} &= \frac{2}{\sqrt{\pi}} \int_0^x e^{-t^2} dt \\
    &= \frac{2}{\sqrt{\pi}} \int_0^x (1 - t^2 + \frac{t^4}{2!} - \frac{t^6}{3!} + \hdots) dt \\
    &= \frac{2}{\sqrt{\pi}} (x - \frac{x^3}{3} + \frac{x^5}{5 \cdot 2!} - \frac{x^7}{7 \cdot 3!} + \hdots)
  \end{split}
\ee
By the look of it, we can say that the series converges for any x because the denominator tends to infinity much faster than the numerator ever could.
However, the convergence of the series is small for large x.
As a matter of fact, for large x, as x $\rightarrow \infty$, erf(x) $\rightarrow$ 1.
So suppose we wanted to know what erf(x) is for large x.
Using the fact that erf(x) $\rightarrow$ 1 as x $\rightarrow \infty$, we could write
\be
1 = \frac{2}{\sqrt{\pi}} \int_0^{\infty} e^{-t^2} dt
\ee
and if we break up the integrals
\be
1 = \frac{2}{\sqrt{\pi}} \int_0^x e^{-t^2} dt + \frac{2}{\sqrt{\pi}} \int_x^{\infty} e^{-t^2} dt
\ee
and rearrange terms, we get
\be \label{eq:erfc}
1 - \frac{2}{\sqrt{\pi}} \int_0^x e^{-t^2} dt = 1 - \text{erf(x)} = \frac{2}{\sqrt{\pi}} \int_x^{\infty} e^{-t^2} dt = \text{erfc(x)}
\ee
where we have an integral from large x to infinity, which happens to be called the complementary error function or erfc(x).
We can try to find an expression for the complementary error function by considering using integration by parts.
Now, you may look at the integrand and think to yourself that there are no two functions that you can use for integration by parts.
But a neat little trick that is often used is to consider multiplying the integrand by the one function.
\be
  \int_x^{\infty} e^{-t^2} dt = \int_x^{\infty} e^{-t^2} \cdot 1 dt = \int_x^{\infty} e^{-t^2} \cdot \frac{-2t}{-2t} dt
\ee
and if we consider,
\be
  \begin{split}
    U &= - \frac{1}{2t} \qquad dV = - 2t e^{-t^2} dt \\
    dU &= \frac{1}{2t^2} dt \qquad V = e^{-t^2}
  \end{split}
\ee
our integral becomes
\be
  \begin{split}
    \int_x^{\infty} e^{-t^2} dt &= \frac{e^{-t^2}}{2t} \Big|_x^{\infty} - \int_x^{\infty} \frac{e^{-t^2}}{2t^2} dt \\
    &= \frac{e^{-x^2}}{2x} - \int_x^{\infty} \frac{e^{-t^2}}{2t^2} dt
  \end{split}
\ee
We can integrate by parts again by using the same trick of multiplying the integrand by the one function
\be
  \int_x^{\infty} e^{-t^2} dt = \frac{e^{-x^2}}{2x} - \int_x^{\infty} \frac{e^{-t^2}}{2t^2} \cdot \frac{-2t}{-2t} dt
\ee
and considering
\be
  \begin{split}
    U &= - \frac{1}{4t^3} \qquad dV = - 2t e^{-t^2} dt \\
    dU &= \frac{3}{4t^4} dt \qquad V = e^{-t^2}
  \end{split}
\ee
to get
\be
  \begin{split}
    \int_x^{\infty} e^{-t^2} dt &= \frac{e^{-x^2}}{2x} - \int_x^{\infty} \frac{e^{-t^2}}{2t^2} \\
    &= \frac{e^{-x^2}}{2x} + \frac{e^{-t^2}}{4t^3} \Big|_x^{\infty} + \int_x^{\infty} \frac{3e^{-t^2}}{4t^4} dt \\
    &= \frac{e^{-x^2}}{2x} - \frac{e^{-x^2}}{4x^3} + \int_x^{\infty} \frac{3e^{-t^2}}{4t^4} dt
  \end{split}
\ee
You can see that we can use this trick each time we integrate by parts and if we integrated an infinite number of times, we'd have an exact expression.
If we go back to the equation \ref{eq:erfc} and solve for erf(x), we can see that the result after n integrations by parts of the complementary error function gives
\be
  \begin{split}
    \text{erf(x)} = 1 - \frac{2}{\sqrt{\pi}} \int_x^{\infty} e^{-t^2} dt = 1 - \frac{2}{\sqrt{\pi}} e^{-x^2} \Big[ \frac{1}{2x} - \frac{1}{2^2 x^3} & + \frac{1 \cdot 3}{2^3 x^5} - \frac{1 \cdot 3 \cdot 5}{2^4 x^7} + \hdots + (-1)^{n-1} \frac{1 \cdot 3 \cdot 5 \cdots (2n-3)}{2^n x^{2n-1}} \Big] \\
    & + (-1)^n \frac{1 \cdot 3 \cdot 5 \cdots (2n-1)}{2^n} \frac{2}{\sqrt{\pi}} \int_x^{\infty} \frac{e^{-t^2}}{t^{2n}} dt
  \end{split}
\ee
where the last term (the term after the brackets, including the integral) is the "remainder" that makes the expression exact.
If we only pay attention to the terms inside the brackets, we can see that these terms form the beginning of divergent series.
Since the numerator grows as (2n+1)!, there is no value of x that would make the denominator grow larger in size than the numerator so if we were to form an infinite series, the series would diverge to $\infty$.
It is because of this fact that physicists would call the series in brackets an \textbf{asymptotic series} if the summation of the terms continued to infinity.
To be formal, if we have a function whose series expansion is
\be
S(x) = c_0 + \frac{c_1}{x} + \frac{c_2}{x} + \hdots
\ee
where the c$_n$'s are just constants, we can say that the partial sum $\sum\limits_{i=0}^n c_i / x^i$ is an asymptotic series expansion of f(x) if the following holds
\be
  \lim_{x \to \infty} x^n \Big[ f(x) - \sum_{i=0}^n \frac{c_i}{x^i} \Big] = 0
\ee
Therefore, for some value n and for arbitrarily large x, the series $S_n(x)$ gives a good approximation to f(x).
If, however, we have a function whose series expansion is
\be
  S(x) = c_0 + c_1x + c_2x^2 + \hdots
\ee
where the c$_n$'s are again just constants, we can say that the partial sum  $\sum\limits_{i=0}^n c_i x^i$ is an asymptotic series expansion of f(x) if the following holds
\be
  \lim_{x \to \infty} \frac{1}{x^n} \Big[ f(x) - \sum_{i=0}^n c_i x^i \Big] = 0
\ee
Therefore, for some value n and for arbitrarily small x, the series $S_n(x)$ gives a good approximation to f(x).

Mathematicians define an asymptotic series expansion as a series whose partial sums give a good approximation to a function of a variable x for arbitrarily small or large values of x.
This definition makes it sound like a convergent Taylor series falls under the category of an asymptotic expansion.
Because of this, it is common in Physics for the term 'asymptotic expansion' to imply a divergent series.
To be mathematically correct, an asymptotic series can be divergent or convergent but, in most cases of interest, asymptotic series never converge.
So it is useful to consider the physicist's perspective when distinguishing between a convergent series and an asympotitc series.
We can go back to the error function erf(x) to show how an asymtptic series can be different from a convergent series.
\be
  \text{erf(x)} = \frac{2}{\sqrt{\pi}} \int_0^x e^{-t^2} dt
\ee
Just as we did in equation 2, if we taylor expand the integrand about $t=0$, and integrate the series term by term, we get the following convergent series,
\be
  \text{erf(x)} = \frac{2}{\sqrt{\pi}} \Big( x - \frac{1}{3}x^3 + \hdots + \frac{(-1)^n}{(2n+1)n!} x^{2n+1} + \hdots \Big)
\ee
which is convergent for all values of x.
And if we recall from earlier, we obtained an asymptotic series of the erf(x)
\be
  \text{erf(x)} \approx 1 - \frac{2}{\sqrt{\pi}} e^{-x^2} \Big[ \frac{1}{2x} - \frac{1}{2^2 x^3} + \frac{1 \cdot 3}{2^3 x^5} - \frac{1 \cdot 3 \cdot 5}{2^4 x^7} + \hdots + (-1)^{n-1} \frac{1 \cdot 3 \cdot 5 \cdots (2n-3)}{2^n x^{2n-1}} \Big] + \text{remainder}
\ee
Now, when $x=3$, we need the first 30 terms of the taylor series to approximate erf(3) to an accuracy of $10^{-5}$, whereas we only need about the first two terms of the asymptotic series (excluding the remainder of course) to obtain the same approximation.
This should not be of any surprise to us because I mentioned earlier in the lecture that the convergence of the Taylor series is slow for (arbitrarily) large x.
If we were to make a plot of the remainder as a function of the number of terms we decide to keep in our series approximation for fixed x
%==============================================================================%
%  Add plot of remainder as a function of M in notes.
%==============================================================================%
we would see that as we keep adding more terms, our approximation gets better and better up to a certain limit, which defines the asymptoticity of the function, and then our approximation becomes worse and worse.
This should make sense because we are trying to use a divergent series to define the value of an integral.

Note, if we use the physics definition of asymptotic series then we have to be clear in saying that not all functions have an asymptotic expansion.
If a function does have such an expansion then that expansion is unique.
However, one can get different asympotitc series expansions for the same function when we consider x to be a complex number z and the convergence of a series to be defined by the radius of a disk on the complex plane.
This is known in complex analysis as Stokes phenomenon but we will not dive into the topic.

\subsubsection*{Saddle Point Method}
Consider the integral
\be
  \int_0^{\infty} F(x) dx = \int_0^{\infty} e^{f(x)} dx
\ee
where the integral of F(x) is dominated by a narrow region in x.
A example plot of F(x) as a function of x would be
%==============================================================================%
%  Add plot of remainder as a function of M in notes.
%==============================================================================%
If we perform a Taylor expansion of f(x) and decide to expand around the point x$_o$ we get
\be
  F(x) \approx \text{exp} \Big[ f(x_o) + (x - x_o)f'(x_o) + \frac{(x - x_o)^2}{2}f''(x_o) \Big]
\ee
If we let x$_o$ to be the maximum of f(x), we can identify that $f'(x_o) = 0$ so our integral becomes
\be
  \begin{split}
    \int_0^{\infty} F(x) dx &\approx \int_0^{\infty} \text{exp} \big[ f(x_o) + \frac{(x - x_o)^2}{2}f''(x_o) \Big] dx \\
    &= \text{exp}[f(x_o)] \int_0^{\infty} \text{exp} \Big[ \frac{(x - x_o)^2}{2}f''(x_o) \Big] dx
  \end{split}
\ee
which is just the integral of a gaussian exp $[A(x-x_o)^2]$ where $A = \frac{f''(x_o)}{2}$.
Using a u-substitution $u = x - x_o$ and $du = dx$, you should be able to get
\be
  \begin{split}
      \int_0^{\infty} F(x) dx &\approx \text{exp}[f(x_o)] \int_0^{\infty} \text{exp} \Big[ \frac{(x - x_o)^2}{2}f''(x_o) \Big] dx \\
      &= \text{exp}[f(x_o)] \Big( \frac{-2 \pi}{f''(x_o)} \Big) ^{\frac{1}{2}}
  \end{split}
\ee
\end{document}

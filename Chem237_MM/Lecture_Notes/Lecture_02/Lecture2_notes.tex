\documentclass{article}
%==============================================================================%
%	                          Packages                                     %
%==============================================================================%
% Packages
\usepackage[utf8]{inputenc}
\usepackage{graphicx}
\usepackage{amsmath}
\usepackage{amssymb}
\usepackage{braket}
\usepackage[margin=0.7in]{geometry}
\usepackage[version=4]{mhchem}
\usepackage{url}
%==============================================================================%
%                           User-Defined Commands                              %
%==============================================================================%
% User-Defined Commands
\newcommand{\be}{\begin{equation}}
\newcommand{\ee}{\end{equation}}
\newcommand{\benum}{\begin{enumerate}}
\newcommand{\eenum}{\end{enumerate}}
\newcommand{\pd}{\partial}
\newcommand{\dg}{\dagger}
%==============================================================================%
%                             Title Information                                %
%==============================================================================%
\title{Chem237: Lecture 2}
\date{4/4/18}
\author{Shane Flynn, Moises Romero}
%==============================================================================%
%==============================================================================%
\begin{document}
\maketitle
\section*{ODE; Variable Coefficients}
We can now consider ODEs with variable coefficients.
Unfortunately most variable coefficient ODEs cannot be solved analytically.

\subsection*{Variable Coefficients; Homogenous}
\be \label{eq:vc_homo}
\frac{dx}{dt} + a(t) x = 0
\ee
Cosider the linear, first order variable coefficient ODE above. 
This example is trivial because the problem is seperable, and can easily be solved by direct integration.
\be
\begin{split}
    \frac{dx}{dt} + &a(t) x = 0 \\
    \int dx \frac{1}{x} &= -\int dt \text{ a}(t) \\
    \ln(x) -\ln(x_0)  &= -\int dt \text{ a}(t) \\
    \ln(x)  = \ln(x_0) &-\int dt \text{ a}(t) \\
    x(t) = x_0 &\exp\left[-\int_{t_0}^t dt \text{ a}(t)\right] + c
\end{split}
\ee
The last line represents the homogeneous solution to the first order variable coefficient ODE (Equation \ref{eq:vc_homo}).
Although the final answer is in terms of an integral, an integral is in general easier than a differential equation to solve.
We therefore call this a solution (because we have made some progress).

\subsubsection*{Differentials and Algebra}
A quick side note that is worth mentioning.
Solving a seperable differential equations (as we did above) seems to imply that the differential operators can be treated algebraically (it looks like we just multiplied by dt in the above example).
Although this is convenient it is not `legit', and we should probably see how it works for the special case of seperable equations.
Consider a simple differential equation
\be \label{eq:dif_alg}
\begin{split}
    f(y) \frac{dy}{dx} &= g(x)\\
    \int dx f(y) \frac{dy}{dx} &= \int dx g(x)
\end{split}
\ee
If we let F(y) define the anti-derivitive of f(y), meaning
\be
F(y) = \int f(y) dy \qquad \Rightarrow \qquad F'(y) = f(y)
\ee
Applying the chain rule to F(y) we find
\be
\frac{d}{dx} F(y) = f(y) \frac{dy}{dx}
\ee
We can now substitute in the chain rule into Equation \ref{eq:dif_alg} and we see 
\be
\begin{split}
    \int dx F'(y) &= \int dx g(x)\\
    F(y) &= \int dx g(x) \\
    \int f(y) dy &= \int dx g(x)
\end{split}
\ee
So we are not actually treating the operators algebraically, we can naturally re-arrange the equation through the chain rule.
This result is true for the seperable equations.
Although most people will solve seperable ODEs by pretending the operators can be manipulated algebraically, it is good to realize why this works in this context (you should not assume you can treat operators algebraically). 

\subsection*{Variable Coefficients; Non-Homogenous}
Now consider the non-homonogeous variable coefficient differential equation
\be \label{eq:vc_non}
\frac{dx}{dt} + a(t) x = f(t)
\ee
Where f(t) is some known (given) function.
The general solution to any non-homogenous differential equation is the summation of associated homogenous equations solution, and a particular solution (x$_p$).

We can therefore use the solution we found to Equation \ref{eq:vc_homo}, for our homogenous solution, and then construct the solution to Equation \ref{eq:vc_non}.
\be
\begin{split}
    x(t) &= x_p(t) + x_0 e^{-A(t)}\\
    A(t) & \equiv \int_{t_0}^t dt \text{ a}(t)
\end{split}
\ee

We now need to find a solution to the particular solution, and unfortunately there is no general strategy for this task.
A decent guess for the form of the particular solution is an expotential function. 
Unfortunately the homogenous solution is already an expotential, and the particular solution needs to tell us something new (it cannot just be the same form as the general solution). 
A decent guess for this case would be an expotential function multiplied by some other function (it could be anything) and we try to solve for this mysterious function.
So we hope that some generic g(t) function exists and we try to solve for it.

So we will guess our particular solution is of the form
\be
x_p(t) = e^{-A(t)}g(t)
\ee
Substituting our guess into Equation \ref{eq:vc_non} we can try to solve for g(t).
\be
\begin{split}
    \frac{dx_p(t)}{dt} + a(t) x_p(t) &= f(t) \\
    \frac{d}{dt} e^{-A(t)}g(t) + a(t)  e^{-A(t)}g(t) &= f(t)\\
    e^{-A(t)} \frac{d}{dt} g(t) + g(t) \frac{d}{dt} e^{-A(t)} + a(t)  e^{-A(t)}g(t) &= f(t)\\
    e^{-A(t)} \frac{d}{dt} g(t) -a(t) e^{-A(t)} g(t) + a(t)  e^{-A(t)}g(t) &= f(t)\\
    e^{-A(t)} \frac{d}{dt} g(t) &= f(t) \Rightarrow \\
    \int^t dt \text{ }e^{A(t)} f(t) &= g(t) \\
\end{split}
\ee
So it appears that we can in general find this mysterious g(t) function, the catch is that we need to compute an integral. 
And we have no guarantee that we can analytically solve that integral. 

\section*{NonLinear ODE}
Let's now consider a more complicated problem, the class of Non-Linear Ordinary Differential Equations.
A theme across all of Differential Equations is that a general strategy for solving differential equations does NOT exist. 
However, there are special cases that can be solved analytically, some of which we will explore here. 

Consider first a somewhat generalized form of the first order ODE
\be
\begin{split}
    a(x) dx + b(y) dy &= 0 \Rightarrow\\
    b(y) \frac{dy}{dx} + a(x) &= 0
\end{split}
\ee
These two lines are equivalent, but the second line is more convenient to work with.
Notice it is separable and can be solved with direct integration.
This is one of the simplest forms of a non-linear equation, (notice there isn't a y dependence in every term, therefore it is non-linear). 

We can generalize the above ODE further, consider
\be \label{eq:gen_ode}
A(x,y)dx + B(x,y)dy = 0
\ee

We will now introduce a well-known topic, the exact differntial (U). 
Exact differentials have some useful properties such as
\be
\begin{split}
    dU &= \frac{\pd U}{\pd x} dx + \frac{\pd U}{\pd y} dy \\
    dU(x,y) &\equiv A dx + B dy \\
    \frac{\pd ^2 U}{\pd x \pd y} &= \frac{\pd ^2 U}{\pd y \pd x}
\end{split}
\ee
Please note: in Line 2 we simply give our partial derivitives names (A and B), no magic occuring here. 

If we assumed our generalized ODE (Equation \ref{eq:gen_ode}) is an exact differential we could write 
\be
U(x,y) + c = 0
\ee
Then the problem becomes what that exact differential is.
So IF we have an exact differential we can solve the problem, and we have some conditions that must be true (relating the partial derivitive orders) if we have an exact differential.\\

As an example consider
\be
(2x+y)dx + (x+3y^2)dy = 0
\ee
We can assume this is an exact differential and construct the partial derivitives.
\be
\begin{split}
    dU & = (2x+y)dx + (x+3y^2)dy = 0\\
    \text{Let } A(x,y) &\equiv \frac{\partial U}{\partial x} = 2x+y  \\
\frac{\partial A}{\partial y} &= 1 \\
    \text{Let} B(x,y) &\equiv \frac{\partial U}{\partial y}  = x+ 3y^2  \\
\frac{\partial B}{\partial x} &= 1 \\
\end{split}
\ee
To see if this is a total differential check to see if the order of differentation changes the final result.
\be
\begin{split}
    \frac{\pd ^2 U}{\pd x \pd y} &= \frac{\pd }{\pd x}\left[\frac{\pd U}{\pd y}\right]= \frac{\pd}{\pd x}\left(x+3y^2\right) = 1 \\
    \frac{\pd ^2 U}{\pd y \pd x} &= \frac{\pd }{\pd y}\left[\frac{\pd U}{\pd x}\right]= \frac{\pd}{\pd y}\left(2x+y\right) = 1 \\
\end{split}
\ee
Clearly the order of differentation does not matter, therefore dU is a total differential.
We can solve for U by simply integrating dU. 
\be
\begin{split}
    dU & = (2x+y)dx + (x+3y^2)dy = 0\\
    \int dU & = \int (2x+y)dx + \int (x+3y^2)dy = 0\\
    U & = x^2 + yx + xy + y^3 + c = 0
\end{split}
\ee

So the solution to the given differential equation is 
\be
x^2 + 2xy + y^3 + c = 0
\ee

\subsection*{Integrating Factor}
What if we don't have a total differential (and cannot use the method shown above then)?
One method for addressing a non-linear ODE is the \textbf{Integrating Factor} (you have probably encountered this in Thermodynamics). 
Consider 
\be
dU = Adx + Bdy
\ee
Such that it is \textbf{NOT} an exact differential.

We can introduce the \textbf{Integrating Factor} $\lambda$(x,y), such that our previous inexact differential becomes an exact differential. 
\be
dU = \lambda(Adx + Bdy)
\ee

We therefore need to find this magical $\lambda$ such that the inexact differential becomes exact.
A theorem exists stating that an integrating factor always exists, so in theory this strategy will always work (in practice you may not be able to solve for $\lambda$). 
For those interested the proof of the theorm can be found at \url{https://proofwiki.org/wiki/Existence_of_Integrating_Factor}. 

$\lambda$ always exists, unfortunately there is no general method for finding the integrating factor.

\subsubsection*{Method}
There are some known forms for A and B that we can recognize to solve for the integrating factor. As an example consider:
\be
dy + f(x) y dx = g(x) dx
\ee

We can always consider an integrating factor (it always exists)
\be
\lambda \left[dy + f(x) y dx\right] = \lambda \left[g(x) dx\right]
\ee

The motivation for using integrating factor is to convert the LHS of the differential equation into the same form as the product rule. 
This trick only works if the RHS is a function of x only, we therefore need to assume the integrating factor is only a function of x ($\lambda \rightarrow \lambda(x)$). 

So we multiply everything by the integrating factor.
\be \label{eq:int_f}
\lambda dy + \lambda f(x) y dx = \lambda g(x) dx
\ee
We want the LHS to be of the form (product rule)
\be
(\lambda y)' = \lambda y' + y\lambda'
\ee
Comparing to Equation \ref{eq:int_f} we assume the following
\be
\lambda' = \lambda(x) f(x)
\ee
This result is seperable and can be evaluated directly to solve for the integrating factor. 
\be
\begin{split}
    \int \frac{d\lambda}{\lambda} &= \int dx f(x) \Rightarrow\\
    \lambda(x) &= \exp\left[\int dx f(x)\right]
\end{split}
\ee
If we are able to find the integrating factor, we can convert our original differential equation into an exact differential.
We can now check to make sure we have an exact differential.
\be
\begin{split}
    dU &\equiv \lambda dy + \lambda f(x)ydx\\
    \frac{d}{dx}\left[\lambda\right] &= \lambda' = \lambda f(x)\\
    \frac{d}{dy} \left[f(x)y\right] &= \lambda f(x)\\
\end{split}
\ee
And we have show this is an exact differential (with our definition of $\lambda'$). 
We can easily integrate to find our final solution
\be
\begin{split}
    dU &\equiv \lambda(x) dy + \lambda(x) f(x)ydx\\
    \int dU &= \int \lambda(x) dy + \int \lambda(x) f(x)ydx\\
    U(x,y) &= \lambda y + y\int \lambda' dx\\
    U(x,y) &= \lambda y + y \lambda + c\\
    U(x,y) &= 2\lambda y + c
\end{split}
\ee

\subsubsection*{Application}
To see how the method works, consider
\be \label{eq:if_ex}
\begin{split}
    xy' + (1+x) y &= e^x\\
    y' + \left(\frac{1+x}{x}\right)y &= \frac{e^x}{x} \Rightarrow \\
    \lambda(x) &= \exp\left[ \int dx \left(\frac{1+x}{x}\right) \right] = xe^x
\end{split}
\ee

Now if we multiply the LHS and RHS of Equation \ref{eq:if_ex} by $\lambda$ and integrate we find
\be
\begin{split}
	dU &= \lambda\left[y' + \left(\frac{1+x}{x}\right)y\right] = \lambda\frac{e^x}{x} \\
    &= xe^x\left[y' + \left(\frac{1+x}{x}\right)y\right] = xe^x\frac{e^x}{x} \\
    \int dU &= \int xe^x\left[y' + \left(\frac{1+x}{x}\right) y\right] = \int dx e^{2x}\\
    U &= \int xe^x dy + \int  xe^x\left(\frac{1+x}{x}\right) y dx = \int dx e^{2x}\\
    U &=  xe^x y + y  xe^x = \frac{e^{2x}}{2} + c\\
	\Rightarrow \quad y &= \frac{1}{2}\left[\frac{e^x}{2x} + \frac{c}{xe^x}\right]
\end{split}
\ee

So we see that a non-linear first order ODE can be solved using the integrating factor method.
If we have an equation of the form
\be
y' + f(x)y = g(x)
\ee
Than the integrating factor uses a non-trivial transformation (assuming a form similar to the product rule) to generate a simple solution in the form of a total differential. 

\subsection*{Homogenous Functions}
We can now look at a different family of equations, homogenous functions.
Consider the following equation
\be
A(x,y)dx + B(x,y)dy = 0
\ee
Assuming both A and B are homogeneous functions of degree r.

\subsubsection*{Recall}
A homogeneous function obeys the following
\be
A(cx,cy) = c^r A(x,y)
\ee

As an example
\be
\begin{split}
    A = x^2 + yx, &\qquad B = y^2\\
    A(cx,cy) = (cx)^2 + (cy)(cx), &\qquad B(cy) = (cy)^2\\
    A(cx,cy) = c^2\left(x^2 + yx\right) = c^2A(x,y), &\qquad B(cy) = c^2\left(y^2\right) = c^2 B(y)\\
\end{split}
\ee
Both functions in the example are homogeneous functions of degree 2. 

\subsubsection*{Substitution}
It turns out that the following substitution can be used to convert homogenoues functions of degree r into a seperable equation.
\be
y = vx
\ee

As with most methods for differential equations, the motivation for this substituion comes from the answer (it works).
Consider the following example
\be
ydx + (2\sqrt{xy} - x) dy = 0
\ee

In this example r=1, let y=vx and substitute in dy = vdx + xdv (assume an exact differential).
\be
\begin{split}
    0 &= ydx + (2\sqrt{xy} - x) dy\\
    0 &= vxdx + (2x\sqrt{v} - x) (vdx + xdv)\\
    0 &= vxdx + (2x\sqrt{v})(vdx) + (2x\sqrt{v})(xdv) - x(vdx) - x(xdv)\\
    &-2x^2\sqrt{v}dv + x^2dv = vxdx + 2xv^{\frac{3}{2}}dx - xvdx \\
    & \frac{-2\sqrt{v}+1}{2v^{3/2}}dv = \frac{dx}{x}
\end{split}
\ee
This final equation is separable and can be solved with integration, so we see that homogenous functions of the same degree can be seperated with the given substitution!

\subsection*{Final Example}
Another example using a substitution to generate a seperable problem. 
Consider the following equation where a,b,c,e,f,g are all constants.
\be
(ax + by + c) dx + (ex + fy + g dy) = 0
\ee

But ax + by + c is not homogeneous because c is a lone constant (same with g).
Consider x = X + $\alpha$ and y = Y + $\beta$, dx = dX, and dy = dY.
Substitute in these definitions we find
\be
(aX + a\alpha + bY + b\beta + c) dX + (eX + e\alpha + fY + f\beta + g) dY = 0
\ee
We can find $\alpha$ and $\beta$,
\be
\begin{split}
    a\alpha + b\beta + c &= 0 \\
    e\alpha + f\beta + g &= 0 \\
\end{split}
\ee
This generates
\be
(aX + bY)dX + (eX + fY)dY = 0
\ee
Where we have a homogeneous equation, and we can solve using the methods from before.

\end{document}

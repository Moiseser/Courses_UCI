\documentclass{article}
%==============================================================================%
%	                          Packages                                     %
%==============================================================================%
% Packages
\usepackage[utf8]{inputenc}
\usepackage{graphicx}
\usepackage{amsmath}
\usepackage{amssymb}
\usepackage{braket}
\usepackage[margin=0.7in]{geometry}
\usepackage[version=4]{mhchem}
%==============================================================================%
%                           User-Defined Commands                              %
%==============================================================================%
% User-Defined Commands
\newcommand{\be}{\begin{equation}}
\newcommand{\ee}{\end{equation}}
\newcommand{\benum}{\begin{enumerate}}
\newcommand{\eenum}{\end{enumerate}}
\newcommand{\pd}{\partial}
\newcommand{\dg}{\dagger}
%==============================================================================%
%                             Title Information                                %
%==============================================================================%
\title{Chem237: Lecture 2}
\date{4/4/18}
\author{Shane Flynn, Moises Romero}
%==============================================================================%
%==============================================================================%
\begin{document}
\maketitle
\section*{ODE; Variable Coefficients}
We can now consider ODEs with variable coefficients.
Unfortunately most variable coefficient ODEs cannot be solved analytically.

\subsection*{Variable Coefficients; Homogenous}
\be \label{eq:vc_homo}
\frac{dx}{dt} + a(t) x = 0
\ee
Cosider the linear, first order variable coefficient ODE above. 
This example is trivial because the problem is seperable, and can easily be solved by direct integration.
\be
\begin{split}
    \frac{dx}{dt} + &a(t) x = 0 \\
    \int dx \frac{1}{x} &= -\int dt \text{ a}(t) \\
    \ln(x) -\ln(x_0)  &= -\int dt \text{ a}(t) \\
    \ln(x)  = \ln(x_0) &-\int dt \text{ a}(t) \\
    x(t) = x_0 &\exp\left[-\int_{t_0}^t dt \text{ a}(t)\right] + c
\end{split}
\ee
The last line represents the homogeneous solution to the first order variable coefficient ODE (Equation \ref{eq:vc_homo}).
Although the final answer is in terms of an integral, an integral is in general easier than a differential equation to solve.
We therefore call this a solution (because we have made some progress).

\subsubsection*{Differentials and Algebra}
A quick side note that is worth mentioning.
Solving a seperable differential equations (as we did above) seems to imply that the differential operators can be treated algebraically (it looks like we just multiplied by dt in the above example).
Although this is convenient it is not `legit', and we should probably see how it works for the special case of seperable equations.
Consider a simple differential equation
\be \label{eq:dif_alg}
\begin{split}
    f(y) \frac{dy}{dx} &= g(x)\\
    \int dx f(y) \frac{dy}{dx} &= \int dx g(x)
\end{split}
\ee
If we let F(y) define the anti-derivitive of f(y), meaning
\be
F(y) = \int f(y) dy \qquad \Rightarrow \qquad F'(y) = f(y)
\ee
Applying the chain rule to F(y) we find
\be
\frac{d}{dx} F(y) = f(y) \frac{dy}{dx}
\ee
We can now substitute in the chain rule into Equation \ref{eq:dif_alg} and we see 
\be
\begin{split}
    \int dx F'(y) &= \int dx g(x)\\
    F(y) &= \int dx g(x) \\
    \int f(y) dy &= \int dx g(x)
\end{split}
\ee
So we are not actually treating the operators algebraically, we can naturally re-arrange the equation through the chain rule.
This result is true for the seperable equations.
Although most people will solve seperable ODEs by pretending the operators can be manipulated algebraically, it is good to realize why this works in this context (you should not assume you can treat operators algebraically). 

\subsection*{Variable Coefficients; Non-Homogenous}
Now consider the non-homonogeous variable coefficient differential equation
\be \label{eq:vc_non}
\frac{dx}{dt} + a(t) x = f(t)
\ee
Where f(t) is some known (given) function.
The general solution to any non-homogenous differential equation is the summation of associated homogenous equations solution, and a particular solution (x$_p$).

We can therefore use the solution we found to Equation \ref{eq:vc_homo}, for our homogenous solution, and then construct the solution to Equation \ref{eq:vc_non}.
\be
\begin{split}
    x(t) &= x_p(t) + x_0 e^{-A(t)}\\
    A(t) & \equiv \int_{t_0}^t dt \text{ a}(t)
\end{split}
\ee

We now need to find a solution to the particular solution, and unfortunately there is no general strategy for this task.
A decent guess for the form of the particular solution is an expotential function. 
Unfortunately the homogenous solution is already an expotential, and the particular solution needs to tell us something new (it cannot just be the same form as the general solution). 
A decent guess for this case would be an expotential function multiplied by some other function (it could be anything) and we try to solve for this mysterious function.
So we hope that some generic g(t) function exists and we try to solve for it.

So we will guess our particular solution is of the form
\be
x_p(t) = e^{-A(t)}g(t)
\ee
Substituting our guess into Equation \ref{eq:vc_non} we can try to solve for g(t).
\be
\begin{split}
    \frac{dx_p(t)}{dt} + a(t) x_p(t) &= f(t) \\
    \frac{d}{dt} e^{-A(t)}g(t) + a(t)  e^{-A(t)}g(t) &= f(t)\\
    e^{-A(t)} \frac{d}{dt} g(t) + g(t) \frac{d}{dt} e^{-A(t)} + a(t)  e^{-A(t)}g(t) &= f(t)\\
    e^{-A(t)} \frac{d}{dt} g(t) -a(t) e^{-A(t)} g(t) + a(t)  e^{-A(t)}g(t) &= f(t)\\
    e^{-A(t)} \frac{d}{dt} g(t) &= f(t) \Rightarrow \\
    \int^t dt \text{ }e^{A(t)} f(t) &= g(t) \\
\end{split}
\ee
So it appears that we can in general find this mysterious g(t) function, the catch is that we need to compute an integral. 
And we have no guarantee that we can analytically solve that integral. 

\section*{NonLinear ODE}
Let's now consider a more complicated problem, the class of Non-Linear Ordinary Differential Equations.
A theme across all of Differential Equations is that a general strategy for solving differential equations does NOT exist. 
However, there are special cases that can be solved analytically, some of which we will explore here. 

Consider first a somewhat generalized form of the first order ODE
\be
\begin{split}
    a(x) dx + b(y) dy &= 0 \Rightarrow\\
    b(y) \frac{dy}{dx} + a(x) &= 0
\end{split}
\ee
These two lines are equivalent, but the second line is more convenient to work with.
Notice it is separable and can be solved with direct integration.
This is one of the simplest forms of a non-linear equation, (notice there isn't a y dependence in every term, therefore it is non-linear). 

We can generalize the above ODE further, consider
\be \label{eq:gen_ode}
A(x,y)dx + B(x,y)dy = 0
\ee

We will now introduce a well-known topic, the exact differntial (U). 
Exact differentials have some useful properties such as
\be
\begin{split}
    dU &= \frac{\pd U}{\pd x} dx + \frac{\pd U}{\pd y} dy \\
    dU(x,y) &\equiv A dx + B dy \\
    \frac{\pd ^2 U}{\pd x \pd y} &= \frac{\pd ^2 U}{\pd y \pd x}
\end{split}
\ee
Please note: in Line 2 we simply give our partial derivitives names (A and B), no magic occuring here. 

If we assumed our generalized ODE (Equation \ref{eq:gen_ode}) is an exact differential we could write 
\be
U(x,y) + c = 0
\ee
Then the problem becomes what that exact differential is.
So IF we have an exact differential we can solve the problem, and we have some conditions that must be true (relating the partial derivitive orders) if we have an exact differential.\\

As an example consider
\be
(2x+y)dx + (x+3y^2)dy = 0
\ee
We can assume this is an exact differential and construct the partial derivitives.
\be
\begin{split}
    dU & = (2x+y)dx + (x+3y^2)dy = 0\\
    \text{Let } A(x,y) &\equiv \frac{\partial U}{\partial x} = 2x+y  \\
\frac{\partial A}{\partial y} &= 1 \\
    \text{Let} B(x,y) &\equiv \frac{\partial U}{\partial y}  = x+ 3y^2  \\
\frac{\partial B}{\partial x} &= 1 \\
\end{split}
\ee
To see if this is a total differential check to see if the order of differentation changes the final result.
\be
\begin{split}
    \frac{\pd ^2 U}{\pd x \pd y} &= \frac{\pd }{\pd x}\left[\frac{\pd U}{\pd y}\right]= \frac{\pd}{\pd x}\left(x+3y^2\right) = 1 \\
    \frac{\pd ^2 U}{\pd y \pd x} &= \frac{\pd }{\pd y}\left[\frac{\pd U}{\pd x}\right]= \frac{\pd}{\pd y}\left(2x+y\right) = 1 \\
\end{split}
\ee
Clearly the order of differentation does not matter, therefore dU is a total differential.
We can solve for U by simply integrating dU. 
\be
\begin{split}
    dU & = (2x+y)dx + (x+3y^2)dy = 0\\
    \int dU & = \int (2x+y)dx + \int (x+3y^2)dy = 0\\
    U & = x^2 + yx + xy + y^3 + c = 0
\end{split}
\ee

So the solution to the given differential equation is 
\be
x^2 + xy + y^3 + c = 0
\ee
%==============================================================================%
% I am here ==> Shane
%==============================================================================%
\subsection*{Integrating Factor}
What if we don't have a total differential (and cannot use the method shown above then)?
One method for addressing a non-linear ODE is the \textbf{Integrating Factor} (you have probably seen this in Thermodynamics). 
Consider dU = Adx + Bdy, such that it is \textbf{NOT} an exact differntial.
We can introduce the \textbf{Integrating Factor} $\lambda$(x,y), such that our previous inexact differential becomes an exact differential. 
\be
dU = \lambda(Adx + Bdy)
\ee
We therefore need to find this magical $\lambda$ such that the inexact differential becomes exact.
A theorem exists stating that an integrating factor always exists, so in theory this strategy will always work (in practice you may not be able to solve for it). 

\subsubsection*{Proof}
%==============================================================================%
% I am here 7-12-18 shane
% This proof has a wiki look there, this explination is not that clear
% https://proofwiki.org/wiki/Existence_of_Integrating_Factor
%==============================================================================%
The proof for this theorem (that an integrating factor always exists) is as follows:
Consider the above example where dU is not an exact differential. 
\be
\begin{split}
    A(x,y)dx &+ B(x,y)dy = 0 \\
    \frac{dy}{dx} &= -\frac{A}{B}
\end{split}
\ee
Where the general solution to this expression is just some constant C
\be
f(x,y) = C
\ee

Then take a total differential of the general solution. 
\be
\begin{split}
    \frac{\partial f}{\partial x}dx &+ \frac{\partial f}{\partial y} dy = 0 \\
    \frac{dy}{dx} &=  - \frac{\frac{\partial f}{\partial x}}{\frac{\partial f}{\partial y}}\\
\end{split}
\ee
Relating the two we can see :
\be
\begin{split}
   \frac{dy}{dx} &=  - \frac{\frac{\partial f}{\partial x}}{\frac{\partial f}{\partial y}}= -\frac{A}{B}\\
   \frac{\frac{\partial f}{\partial x}}{A} &= \frac{\frac{\partial f}{\partial y}}{B}
\end{split}
\ee
We can then relate this ratio using a factor , which is the integrating factor $\lambda(x,y)$ :
\be
\begin{split}
\frac{\partial f}{\partial x} &= \lambda A  \\
\frac{\partial f}{\partial y} &= \lambda B
\end{split}
\ee
We multiply are general differential equation by an integrating factor as follows :
\be
\lambda A dx + \lambda B dy = 0
\ee
Unfortunately there is no general method for finding it, but $\lambda$ always exists!
There are known forms for A and B that we can recognize to solve for the integrating factor. As an example consider:
\be
y' + f(x) y = g(x) \Rightarrow dy + f(x) y dx = g(x) dx
\ee
We can always consider an integrating factor
\be
\lambda \left[dy + f(x) y dx\right] = \lambda g(x) dx
\ee
The idea is now to integrate each side, however, this will only be useful if the RHS is only a function of x.
If it turns out that $\lambda \rightarrow \lambda(x)$ than the RHS is always a function of x.
If $\lambda$(x,y) than we may not be able to solve the LHS.
So we will assume $\lambda \rightarrow \lambda(x)$ and see if it works.
The whole point of the method is to get an exact differential, therefore assume the LHS is exact.
\be
\frac{\pd\lambda}{\pd x} = \frac{\pd (\lambda f y)}{\pd y} = \lambda(x) f(x)
\ee
This new equation is separable and we can solve by integration.
\be
\int d\lambda \frac{1}{\lambda} = \int dx f(x) \Rightarrow \lambda(x) = \exp\left[\int dx f(x)\right]
\ee
So we see that we can find the integrating factor with $\lambda(x)$.
Now that we have an exact differential we need to confirm
\be
\begin{split}
    dU &\equiv \lambda(dy + f(x)ydx)\\
    \frac{\pd U}{\pd x} &= \lambda\\
    U(x,y) = \lambda(x) y
\end{split}
\ee
So we can solve this problem by integration, and find a general equation for any f(x), g(x). Another example consider
\be
\begin{split}
    xy' + (1+x) y = e^x \Rightarrow y' + \left(\frac{1+x}{x}\right)y &= \frac{e^x}{x}\\
    \lambda(x) = \exp\left[ \int dx \frac{1+x}{x} \right] = xe^x
\end{split}
\ee
Now if we multiply the LHS and RHS by $\lambda$
\be
\int xe^x\left[y' + \frac{1+x}{x} y\right] = \int dx e^{2x} \Rightarrow U(x,y) = xe^xy = \frac{1}{2} e^{2x} + c \Rightarrow y = \frac{e^x}{2x} + \frac{c}{x} e^{-x}
\ee

So we see that a non-linear first order ODE can be solved with the integrating factor method.
If we have an equation of the form
\be
y' + f(x)y = g(x)
\ee
Than the integrating factor is a useful approach, using a non-trivial transformation to generate a simple solution.

\subsection*{Homogenous Functions}
Consider the following equation
\be
A(x,y)dx + B(x,y)dy = 0
\ee
Assume both A and B are homogeneous functions of degree r.

\subsubsection*{Recall}
A homogeneous function obeys the following
\be
A(cx,cy) = c^r A(x,y)
\ee

As an example
\be
\begin{split}
    A = x^2 + yx, &\qquad B = y^2\\
    A(cx,cy) = (cx)^2 + (cy)(cx), &\qquad B(cy) = (cy)^2\\
    A(cx,cy) = c^2\left(x^2 + yx\right) = c^2A(x,y), &\qquad B(cy) = c^2\left(y^2\right) = c^2 B(y)\\
\end{split}
\ee
Both functions in the example are homogeneous functions of degree 2. 

For this type of equation for we can generate a separable equation through the substitution
\be
y = vx
\ee
Consider our example above we find
\be
ydx + (2\sqrt{xy} - x) dy = 0
\ee
In this example r=1, let y=vx and substitute in dy = vdx + xdv (assume an exact differential).
\be
vxdx + (2x\sqrt{v} - x) (vdx + xdv) = 0 \Rightarrow \frac{2\sqrt{v}-1}{2v^{3/2}}dv = -\frac{1}{x} dx
\ee
And this equation is separable and can be solved with integration!

\subsection*{Final Example}
Another example of the same class of problem.
\be
(ax + by + c) dx + (ex + fy + g dy) = 0
\ee
Where a,b,c,e,f,g are all constants.
But ax + by + c is not homogeneous because c is a lone constant (same with g).
Consider x = X + $\alpha$ and y = Y + $\beta$, dx = dX, and dy = dY.
Substitute in these definitions we find
\be
(aX + a\alpha + bY + b\beta + c) dX + (eX + e\alpha + fY + f\beta + g) dY = 0
\ee
We can find $\alpha$ and $\beta$,
\be
\begin{split}
    a\alpha + b\beta + c &= 0 \\
    e\alpha + f\beta + g &= 0 \\
\end{split}
\ee
This generates
\be
(aX + bY)dX + (eX + fY)dY = 0
\ee
Where we have a homogeneous equation, and we can solve using the methods from before.

\end{document}

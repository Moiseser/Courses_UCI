\documentclass{article}
%==============================================================================%
%	                          Packages                                     %
%==============================================================================%
% Packages
\usepackage[utf8]{inputenc}
\usepackage{graphicx}
\usepackage{amsmath}
\usepackage{amssymb}
\usepackage{braket}
\usepackage[margin=0.7in]{geometry}
\usepackage[version=4]{mhchem}
%==============================================================================%
%                           User-Defined Commands                              %
%==============================================================================%
% User-Defined Commands
\newcommand{\be}{\begin{equation}}
\newcommand{\ee}{\end{equation}}
\newcommand{\benum}{\begin{enumerate}}
\newcommand{\eenum}{\end{enumerate}}
\newcommand{\pd}{\partial}
\newcommand{\dg}{\dagger}
%==============================================================================%
%                             Title Information                                %
%==============================================================================%
\title{Chem237: Lecture 3}
\date{4/5/18}
\author{Shane Flynn,Moises Romero, Alan Robledo}
%==============================================================================%
%	Everyone Please Make Comments if Something Needs to be Reviewed        %
%                           Or just fix it yourself!                           %
%==============================================================================%
\begin{document}
\maketitle
\section*{Stuff}
% Everything before series section is the last of the ODE chapter.
\section*{Series}
Chapter 2 discusses tests for convergence of series and methods to determine the sum of a series.

Any ordered list of elements can be considered a \textbf{sequence}.
If we have a sequence of numbers, real or complex, this could be written as
\be
  a_1, a_2, \hdots, a_n
\ee
where n is the total amount of numbers (elements) in the sequence.

A \textbf{series} is defined as the summation of a sequence of numbers
\be
S=\sum_{n=1}^{\infty} a_n
\ee
A \textbf{partial sum} is defined as the summation of parts of sequence:
\be
S_N = \sum_{n=1}^{N} a_n
\ee
So it goes without saying that the infinte limit of a partial sum defines the infinite sum.
\be
S = \lim_{N \to \infty}{S_N} = \lim_{N \to \infty} \sum_{n=1}^{N} a_n
\ee
The value of S determines whether the series is labeled as convergent or divergent.
\subsection*{Convergence and Divergence}
A series $\sum\limits_{n=1}^{\infty} a_n$ is said to be \textbf{convergent} if its partial sums S$_1$, S$_2$, $\hdots$ tend towards some finite number.
If the partial sums tend towards $+\infty$ or $-\infty$, then the series is said to be \textbf{divergent}.
In other words, if
\be
\lim_{n \to \infty}{a_n} = 0
\ee
then the series is convergent.
If the limit is any other value or the limit does not exist, then the series is divergent.

If the series
\be
\sum_{n=1}^{\infty} |a_n|
\ee
converges to some finite value, then we say that $\sum\limits_{n=1}^{\infty} a_n$ is \textbf{absolutely convergent}.
If the series in equation (add number) diverges but $\sum\limits_{n=1}^{\infty} a_n$ is convergent, then we say that $\sum\limits_{n=1}^{\infty} a_n$ is \textbf{conditionally convergent}.

For example, consider the alternating harmonic series:
\be
\sum_{n = 1}^{\infty} \frac{(-1)^{n+1}}{n} = 1 - \frac{1}{2} + \frac{1}{3} - \hdots
\ee
This is an example of an alternating series because each successive term switches between a positive and negative sign.
Note: the $(-1)^{n+1}$ could be $(-1)^{n}$ or any other integer exponent and the series would still be considered an alternating series.
This series is said to be convergent and there are several tests that can be used to prove that this is true, some of which will be discussed later in the lecture.
In the infinite limit, the series converges to
\be
\sum_{n = 1}^{\infty} \frac{(-1)^{n+1}}{n} = \ln(2)
\ee
However, the harmonic series
\be
\sum_{n = 1}^{\infty} \frac{1}{n} = 1 + \frac{1}{2} + \frac{1}{3} + \hdots
\ee
is a well-known divergent series.
Therefore, the alternating harmonic series is not absolutely convergent but it is conditionally convergent.
% Consider the geometric series:
% \be
% S(x) = \sum_{n=0}^\infty x^n
% \ee
% If $x<1$ then this series converges. \\
% If $x>1$ then this series diverges.
Despite the harmonic series being divergent, we can show that the partial sums converge to:
\be
S_N = \sum_{n=1}^{N} \frac{1}{n} \approx \frac{1}{N}
\ee
% Somebody find the proof for above. I'm sure its easy to find.
Another well-known series is the geometric series:
\be
\sum_{n=0}^{\infty} a x^n = a + ax + ax^2 + \hdots
\ee
where a and x can be any real number.
Consider the case where a = 1, so the series and partial sum looks like
\be
\sum_{n=0}^{\infty} x^n = 1 + x + x^2 + \hdots \quad \text{and} \quad S_N = \sum_{n=0}^{N} x^n
\ee
If we wanted to know the value of the partial sum, we would need to know how the partial sums S$_N$ and S$_{N+1}$ are related.
Since we have two unknowns, that means we need two equations.
The first equation comes from noticing that each successive partial sum is equal to the previous partial sum plus the next term in the series.
\be
S_{N+1} = S_N + x^{N+1}
\ee
The other equation comes from noticing that
\be
\begin{split}
1 + x(1 + x + x^2 + \hdots + x^N) &= 1 + x + x^2 + x^3 + \hdots + x^{N+1} \\
1 + x (S_N) &= S_{N+1}
\end{split}
\ee
So we can define a system of linear equations as
\be
\begin{split}
S_{N+1} - S_N &= x^{N+1} \\
1 + xS_N &= S_{N+1}
\end{split}
\ee
and solve for S$_N$ by plugging in S$_{N+1}$ and rearranging to get
\be
S_N = \frac{1-x^{N+1}}{1-x}
\ee
If $|x| < 1$, we can show that the infinte sum converges
\be
S = \lim_{N \to \infty}S_N = \frac{1}{1-x}
\ee
and if $|x|\geq$ 1, then the series diverges.
\subsubsection*{D'Alembert-Lauche test}
Sometimes known as the 'ratio test' , tests the convergence of real numbers. Consider the sequence :
\be
\sum_{n=1}^{\infty}a_n
\ee
If :
\be
\frac{a_{n+1}}{a_n} < 1
\ee
The series \textbf{converges absolutely}. If $>1$ diverges.
If it $=1$ , then we must choose a different method.
\subsubsection*{Integral Test for Convergence}
This test is used to test non-negative terms for convergence., consider a sequence of continuous function $f(n)$
\be
\sum_{n=1}^{\infty} f(n)
\ee
with :
\be
\int_{1}^{\infty}f(x)dx
\ee
\subsubsection*{Reinman Zeta Function}
\be
\mathcal{L}(s) = 1 + \frac{1}{2^s} + \frac{1}{3^s}+ \hdots = \sum_{n=1}^{\infty} = \frac{1}{n^s}
\ee
Where s is an integer of a simple real variable ...
Consider a real s.
\be
\begin{split}
\frac{a_{n+1}}{a_n}= \left(\frac{n}{n+1}\right)^s {n \to \infty}:  \left(1+\frac{1}{n}\right)^{-s} = \left( 1 - \frac{s}{n}\right)+ \hdots \\
\left(1 + \frac{1}{n}\right)^{-s} = \exp\left[{\ln{\left(1+\frac{1}{n}\right)^{-s}}}\right] = \exp\left[-s*{\ln{\left(1+\frac{1}{n}\right)^{}}}\right] = \exp\left[(-s)*\frac{1}{n}\right]=1 - \frac{s}{n}
\end{split}
\ee
Since this doesn't help us determine whether it converges or diverges we will try the Integral test.
Consider the Reinman zeta function defined above as a function of x :
\be
f(x) = \frac{1}{x^s}
\ee
We will now do the integral test :
\be
\begin{split}
    \int_1^\infty \frac{1}{x^s} = \left(\frac{1}{1-s}\right)\left(\frac{1}{x^{1-s}}\right) \\
    \text{If } s>1  \text{ it converges} \\
    \text{If } s<1 \text{ it diverges} \\
\end{split}
\ee
These rules for s can then be applied to the results of the ratio test.
\subsubsection*{Alternating Series}
If a series is not absolute convergent :
\be
\sum_n a_n = a_n = (-1)^n |a_n|
\ee
Consider  the series :
\be
S(+1) = S = \sum_{n=1}^\infty \frac{(-1)^{n-1}}{n} = 1 - \frac{1}{2} + \frac{1}{3} - \hdots = \ln(2)
\ee
\be
S(x) = \sum_{n=1}^\infty \frac{x^{n-1}}{n} = \ln(1+x)
\ee





\end{document}

\documentclass{article}
%==============================================================================%
%	                          Packages                                     %
%==============================================================================%
% Packages
\usepackage[utf8]{inputenc}
\usepackage{graphicx}
\usepackage{amsmath}
\usepackage{amssymb}
\usepackage{braket}
\usepackage[margin=0.7in]{geometry}
\usepackage[version=4]{mhchem}
%==============================================================================%
%                           User-Defined Commands                              %
%==============================================================================%
% User-Defined Commands
\newcommand{\be}{\begin{equation}}
\newcommand{\ee}{\end{equation}}
\newcommand{\benum}{\begin{enumerate}}
\newcommand{\eenum}{\end{enumerate}}
\newcommand{\pd}{\partial}
\newcommand{\dg}{\dagger}
%==============================================================================%
%                             Title Information                                %
%==============================================================================%
\title{Chem237: Lecture 3}
\date{4/5/18}
\author{Shane Flynn,Moises Romero}
%==============================================================================%
%	Everyone Please Make Comments if Something Needs to be Reviewed        %
%                           Or just fix it yourself!                           %
%==============================================================================%
\begin{document}
\maketitle
\section*{Stuff}
\section*{Series}
 Chapter 2 discusses tests for convergence of series and methods to obtain the sum of a series in closed form.
 A series is a sum of various numbers, or elements of a sequence. A sequence is defined by :
 \be
  a_1, a_2, \hdots a_n
 \ee
 Where n is the total number of terms in the sequence.
 $a_n$ can be real or complex.
A series is defined by :
\be
S=\sum_{n=1}^{\infty}a_1 + a_2 \hdots + a_n
\ee
A series is the sum of infinite number of terms possible in the sequence.
A partial sum is the summation of parts of sequence:
\be
S_N = \sum_{n=1}^{N} a_n
\ee
Where capital N is the highest number of the sequence.
With these two definitions we can state that a series 'S' is the limit of a partial sum $S_N$ as N approaches infinity
\be
S = \lim_{N \to \infty}{S_N}
\ee
\subsection*{Convergence and Divergence}
Absolute convergence occurs in a series if :
\be
S= \sum_{n=1}^{\infty} |a_n|
\ee
The absolute value of the sequence equals a finite number.
A necessary condition for convergence is that the sequence approaches 0 as the limit goes to infinity :
\be
\lim_{N \to \infty} a_n = 0
\ee
This means that the end of the sequence equals to 0 in order to get a finite number, and thus convergence.
\bigskip
Consider the series :
\be
S(x) = \sum_{n=0}^\infty x^n
\ee
If $x<1$ then this series converges. \\
If $x>1$ then this series diverges.

A divergent series does not converge to a finite number and neither does the limit.
An example of a divergent series is :
\be
\sum_{n=1}^{\infty} \frac{1}{n}
\ee
However if we consider a partial sum of these sequence it does converge :
\be
S(N) = \sum_{n=1}^{N} \frac{1}{n} \approx \frac{1}{N}
\ee
If a series but diverges if taking the absolute value of the sequence but converges without the absolute value of the entire sequence, this is called \textbf{Conditional Convergence}.
An example of this series is :
\be
\sum_{n=1}^{\infty}\frac{(-1)^n}{n}
\ee
\subsubsection*{Geometric Series}
A geometric series :
A geometric series has a constant ratio between the terms.
\be
\begin{split}
\sum_{n=0}^{\infty}x^n  \\
S_N = \sum_{n=0}^{N} x^n = x^0 +x^1 + x^2 + x^3 \hdots
\end{split}
\ee
Consider the geometric series at 1 term higher which can be described one of two ways :
\be
\begin{split}
S_{N+1} = x^{N+1} + S_N \\
S_{N+1} = xS_N + 1
\end{split}
\ee

Solving the system of linear equation gives you :
\be
S_N = \frac{1+x^N}{1-x}
\ee
Using this equation consider $|x| < 1$ :
\be
S = \lim_{N \to \infty}S_N = \frac{1}{1-x}
\ee
If $|x|\leq$ 1 then the series diverges.
\subsubsection*{D'Alembert-Lauche test}
Sometimes known as the 'ratio test' , tests the convergence of real numbers. Consider the sequence :
\be
\sum_{n=1}^{\infty}a_n
\ee
If :
\be
\frac{a_{n+1}}{a_n} < 1
\ee
The series \textbf{converges absolutely}. If $>1$ diverges.
If it $=1$ , then we must choose a different method.
\subsubsection*{Integral Test for Convergence}
This test is used to test non-negative terms for convergence., consider a sequence of continuous function $f(n)$
\be
\sum_{n=1}^{\infty} f(n)
\ee
with :
\be
\int_{1}^{\infty}f(x)dx
\ee
\subsubsection*{Reinman Zeta Function}
\be
\mathcal{L}(s) = 1 + \frac{1}{2^s} + \frac{1}{3^s}+ \hdots = \sum_{n=1}^{\infty} = \frac{1}{n^s}
\ee
Where s is an integer of a simple real variable ...
Consider a real s.
\be
\begin{split}
\frac{a_{n+1}}{a_n}= \left(\frac{n}{n+1}\right)^s {n \to \infty}:  \left(1+\frac{1}{n}\right)^{-s} = \left( 1 - \frac{s}{n}\right)+ \hdots \\
\left(1 + \frac{1}{n}\right)^{-s} = \exp\left[{\ln{\left(1+\frac{1}{n}\right)^{-s}}}\right] = \exp\left[-s*{\ln{\left(1+\frac{1}{n}\right)^{}}}\right] = \exp\left[(-s)*\frac{1}{n}\right]=1 - \frac{s}{n}
\end{split}
\ee
Since this doesn't help us determine whether it converges or diverges we will try the Integral test.
Consider the Reinman zeta function defined above as a function of x :
\be
f(x) = \frac{1}{x^s}
\ee
We will now do the integral test :
\be
\begin{split}
    \int_1^\infty \frac{1}{x^s} = \left(\frac{1}{1-s}\right)\left(\frac{1}{x^{1-s}}\right) \\
    \text{If } s>1  \text{ it converges} \\
    \text{If } s<1 \text{ it diverges} \\
\end{split}
\ee
These rules for s can then be applied to the results of the ratio test.
\subsubsection*{Alternating Series}
If a series is not absolute convergent :
\be
\sum_n a_n = a_n = (-1)^n |a_n|
\ee
Consider  the series :
\be
S(+1) = S = \sum_{n=1}^\infty \frac{(-1)^{n-1}}{n} = 1 - \frac{1}{2} + \frac{1}{3} - \hdots = \ln(2)
\ee
\be
S(x) = \sum_{n=1}^\infty \frac{x^{n-1}}{n} = \ln(1+x)
\ee





\end{document}

\documentclass{article}
%==============================================================================%
%	                          Packages                                     %
%==============================================================================%
% Packages
\usepackage[utf8]{inputenc}
\usepackage{graphicx}
\usepackage{amsmath}
\usepackage{amssymb}
\usepackage{braket}
\usepackage[margin=0.7in]{geometry}
\usepackage[version=4]{mhchem}
%==============================================================================%
%                           User-Defined Commands                              %
%==============================================================================%
% User-Defined Commands
\newcommand{\be}{\begin{equation}}
\newcommand{\ee}{\end{equation}}
\newcommand{\benum}{\begin{enumerate}}
\newcommand{\eenum}{\end{enumerate}}
\newcommand{\pd}{\partial}
\newcommand{\dg}{\dagger}
%==============================================================================%
%                             Title Information                                %
%==============================================================================%
\title{Chem237: Lecture 3}
\date{4/5/18}
\author{Alan Robledo, Shane Flynn, Moises Romero}
%==============================================================================%
%	Everyone Please Make Comments if Something Needs to be Reviewed        %
%                           Or just fix it yourself!                           %
%==============================================================================%
\begin{document}
\maketitle
\section*{Stuff}
% Everything before series section is the last of the ODE chapter.
\section*{Series}
Chapter 2 discusses tests for convergence of series and methods to determine the sum of a series.
% Needs a better intro.
Any ordered list of elements can be considered a \textbf{sequence}.
If we have a sequence of numbers, real or complex, this could be written as
\be
  a_1, a_2, \hdots, a_n
\ee
where n is the total amount of numbers (elements) in the sequence.

A \textbf{series} is defined as the summation of a sequence of numbers
\be
S=\sum_{n=1}^{\infty} a_n
\ee
A \textbf{partial sum} is defined as the summation of parts of sequence:
\be
S_N = \sum_{n=1}^{N} a_n
\ee
So it goes without saying that the infinte limit of a partial sum defines the infinite sum.
\be
S = \lim_{N \to \infty}{S_N} = \lim_{N \to \infty} \sum_{n=1}^{N} a_n
\ee
The value of S determines whether the series is labeled as convergent or divergent.
\subsection*{Convergence and Divergence}
A series $\sum\limits_{n=1}^{\infty} a_n$ is said to be \textbf{convergent} if its partial sums S$_1$, S$_2$, $\hdots$ tend towards some finite number.
If the partial sums tend towards $+\infty$ or $-\infty$, then the series is said to be \textbf{divergent}.
In other words, if
\be
\lim_{n \to \infty}{a_n} = 0
\ee
then the series is convergent.
If the limit is any other value or the limit does not exist, then the series is divergent.

If the series
\be
\sum_{n=1}^{\infty} |a_n|
\ee
converges to some finite value, then we say that $\sum\limits_{n=1}^{\infty} a_n$ is \textbf{absolutely convergent}.
If the series in equation (add number) diverges but $\sum\limits_{n=1}^{\infty} a_n$ is convergent, then we say that $\sum\limits_{n=1}^{\infty} a_n$ is \textbf{conditionally convergent}.
A very useful property of an absolutely convergent series is that any rearrangement of the terms that make up the original sum will always give the same sum.
On the other hand, any rearrangement of the terms that make up a conditionally convergent series will always give a different result for the sum.

For example, consider the alternating harmonic series:
\be
\sum_{n=1}^{\infty} a_n = \sum_{n = 1}^{\infty} \frac{(-1)^{n+1}}{n} = 1 - \frac{1}{2} + \frac{1}{3} - \hdots
\ee
This is an example of an alternating series because each successive term switches between a positive and negative sign.
Note: the $(-1)^{n+1}$ could be $(-1)^{n}$ or any other integer exponent and the series would still be considered an alternating series.
This series is said to be convergent and there are several tests that can be used to prove that this is true, some of which will be discussed later in the lecture.
In the infinite limit, the series converges to
\be
\sum_{n = 1}^{\infty} \frac{(-1)^{n+1}}{n} = \ln(2)
\ee
A quick way to prove this is to consider the following function
\be
S(x) = \sum_{n = 1}^{\infty} (-1)^{n+1} \frac{x^n}{n}
\ee
If we insert $x = 1$, we get back the alternating harmonic series.
And if you take a look at your list of every conceivable taylor expansion in the universe, you will find
\be
S(x) = \sum_{n = 1}^{\infty} (-1)^{n+1} \frac{x^n}{n} = \ln(1+x)
\ee
Since we know that our original series is equal to S(x) when x = 1, then we can say that the oringial series is equal to ln(2).
% Somebody find a better place to put the proof for ln(2), or add more flow.
However, the harmonic series
\be
\sum_{n=1}^{\infty} |a_n| = \sum_{n = 1}^{\infty} \frac{1}{n} = 1 + \frac{1}{2} + \frac{1}{3} + \hdots
\ee
is a well-known divergent series.
Therefore, from our properties mentioned earlier, we can say that the alternating harmonic series is conditionally convergent and not absolutely convergent.
If we consider rearranging the plus and minus signs in the alternating harmonic series, you can show that the infinite sum does not evaluate to ln(2),
\be
\sum_{n = 1}^{\infty} \frac{(-1)^{n}}{n} = -1 + \frac{1}{2} - \frac{1}{3} + \hdots = -\ln(2)
\ee
which is expected because the series is conditionally convergent.
On a side note, despite the harmonic series being divergent, we can show that the partial sums converge to:
\be
S_N = \sum_{n=1}^{N} \frac{1}{n} \approx \frac{1}{N}
\ee
% Somebody find the proof for above. I'm sure its easy to find.
Another well-known series is the geometric series:
\be
\sum_{n=0}^{\infty} a x^n = a + ax + ax^2 + \hdots
\ee
where a and x can be any real number.
Consider the case where a = 1, so the series and partial sum looks like
\be
\sum_{n=0}^{\infty} x^n = 1 + x + x^2 + \hdots \quad \text{and} \quad S_N = \sum_{n=0}^{N} x^n
\ee
If we wanted to know the value of the partial sum, we would need to know how the partial sums S$_N$ and S$_{N+1}$ are related.
Since we have two unknowns, that means we need two equations.
The first equation comes from noticing that each successive partial sum is equal to the previous partial sum plus the next term in the series.
\be
S_{N+1} = S_N + x^{N+1}
\ee
The other equation comes from noticing that
\be
\begin{split}
1 + x(1 + x + x^2 + \hdots + x^N) &= 1 + x + x^2 + x^3 + \hdots + x^{N+1} \\
1 + x (S_N) &= S_{N+1}
\end{split}
\ee
So we can define a system of linear equations as
\be
\begin{split}
S_{N+1} - S_N &= x^{N+1} \\
1 + xS_N &= S_{N+1}
\end{split}
\ee
and solve for S$_N$ by plugging in S$_{N+1}$ and rearranging to get
\be
S_N = \frac{1-x^{N+1}}{1-x}
\ee
If $|x| < 1$, we can show that the infinte sum converges
\be
S = \lim_{N \to \infty}S_N = \frac{1}{1-x}
\ee
and if $|x|\geq$ 1, then the series diverges.
\section*{Convergence tests}
\subsubsection*{D'Alembert-Lauche test}
Often referred to as the 'ratio test', if you have any series $\sum\limits_{n=1}^{\infty} a_n$, you can check whether it is absolutely convergent or divergent by evaluating the infinite limit of the ratio between a$_{n+1}$ and a$_n$
\be
\lim_{n \to \infty} \Big| \frac{a_{n+1}}{a_n} \Big|
\ee
If the limit is less than 1, then the series is absolutely convergent.
If the limit is greater than 1, then the series is divergent.
If the limit is equal to 1, then the test is inconclusive and another test should be used instead.
\subsubsection*{Integral Test}
Convergence of a series $\sum\limits_{n=1}^{\infty} f(n)$, where f(n) is continuous, can be determined by determining convergence of the definite integral.
\be
\int_{1}^{\infty}f(x)dx
\ee
Notice that the upper and lower limits of the definite integral are the same as the upper and lower limits of the series.
Note: the integral test does not determine whether a series is absolutely or conditionally convergent.

Using these two tests, we can test the Riemann Zeta (RZ) function for convergence.
For those who have not heard of the RZ function, it is defined as
\be
\zeta(s) = \sum_{n=1}^{\infty} \frac{1}{n^s} = 1 + \frac{1}{2^s} + \frac{1}{3^s}+ \hdots
\ee
where s is an integer.
Using the ratio test, we first need to evaluate the limit.
\begin{align}
\lim_{n \to \infty} \Big| \frac{a_{n+1}}{a_n} \Big| &= \lim_{n \to \infty} \frac{\frac{1}{(n+1)^s}}{\frac{1}{n^s}} = \lim_{n \to \infty} \Big( \frac{n}{n+1} \Big) ^s
= \lim_{n \to \infty} \Big( \frac{n+1}{n} \Big) ^{-s} = \lim_{n \to \infty} \Big( 1 + \frac{1}{n} \Big) ^{-s} \\
&= \lim_{n \to \infty} \exp \Big[ \ln \Big( 1 + \frac{1}{n} \Big) ^{-s} \Big] = \lim_{n \to \infty} \exp \Big[ (-s) \ln \Big( 1 + \frac{1}{n} \Big) \Big] \\
\end{align}
We can taylor expand to make the substitution $\ln(1 + \frac{1}{n}) = \frac{1}{n} - \frac{1}{2}(\frac{1}{n})^2 + \hdots \approx \frac{1}{n}$.
\be
\lim_{n \to \infty} \Big| \frac{a_{n+1}}{a_n} \Big| = \lim_{n \to \infty} \exp \Big[ (-s) \ln \Big( 1 + \frac{1}{n} \Big) \Big] \approx \lim_{n \to \infty} \exp \Big[ \frac{-s}{n} \Big]
\ee
Taylor expanding again gives
\be
\lim_{n \to \infty} \exp \Big[ \frac{-s}{n} \Big] = \lim_{n \to \infty} 1 - \frac{s}{n} + \frac{1}{2} \Big( \frac{-s}{n} \Big)^2 + \hdots \approx \lim_{n \to \infty} 1 - \frac{-s}{n} = 1
\ee
which means that the ratio test is inconclusive. So we can try the integral test.
\be
\int_{1}^{\infty} \frac{1}{x^s} dx = \frac{-1}{(s-1)(x^{s-1})} \Big\vert_{1}^{\infty} = \frac{1}{s-1}
\ee
So if s $>$ 1, the series is convergent and if s $<$ 1, the series is divergent.
\subsubsection*{Alternating Series Test}
As the name implies, this test is only to be used when you have an alternating series, e.g. $\sum\limits_{n=1}^{\infty} (-1)^n a_n$.
Any series of this form is considered convergent if:
\begin{enumerate}
  \item $\lim_{n \to \infty} a_n = 0$
  \item $|a_n| > |a_{n+1}|$
\end{enumerate}
If these conditions cannot be met, the series is considered divergent.
Using the alternating harmonic series as an example, we can check the two conditions and see that
\be
\lim_{n \to \infty} \frac{1}{n} = 0 \quad \text{and} \quad \frac{1}{n} > \frac{1}{n+1}
\ee
which shows that the series is convergent, as expected.
However, this test is not enough to show whether an alternating series is absolutely or conditionally convergent.
There are a few other tests that you could use such as the limit comparison test, p-series test, etc, but the three we went over are very useful on their own for testing convergence.
\end{document}

\documentclass{article}
% Packages
\usepackage[utf8]{inputenc}
\usepackage{graphicx}
\usepackage{amsmath}
\usepackage{braket}
\usepackage[margin=0.7in]{geometry}
\usepackage{hyperref}
\usepackage[version=4]{mhchem}
% User-Defined Commands
\newcommand{\be}{\begin{equation}}
\newcommand{\ee}{\end{equation}}
\newcommand{\benum}{\begin{enumerate}}
\newcommand{\eenum}{\end{enumerate}}
\newcommand{\pd}{\partial}
% Title Information
\title{Chem231B: Lecture 3}
\author{Shane Flynn (swflynn@uci.edu)Moises Romero(moiseser@uci.edu)}
\date{1/10/18}

\begin{document}
\maketitle

Cohen-Tannoudji Volume 1 Chapter 5 P.504

\section*{Name}
Last class we connected the HO to the uncertainty principle.
The conclusion is that squeezing a particle into a smaller space requires a higher momentum. 
In Quantum Mechanics the n=0 state is no localized at the bottom of the well, but has a finite distribution. 
This is characterized by the root-mean-square deviation in X as we found last lecture. 

For our HO the average Potential Energy of the particle in the state $\ket{\phi_n}$ is given by
\be
\braket{V(X)} = \frac{1}{2} m\omega^2 \braket{X^2} = \frac{1}{2} m\omega^2 \braket{\Delta X^2}
\ee
Where we know the variance can be replaced for a stationary state as the expectation of X is 0. 
We can repeat the analysis for the average kinetic energy as
\be
\braket{\frac{P^2}{2m}} = \frac{1}{2m}(\Delta P)^2
\ee
Substituting in our expressions from lecture 2 for the variance we see that
\be

% Comments are given  by the % character
\documentclass{article}
% Packages
\usepackage[utf8]{inputenc}
\usepackage{graphicx}
\usepackage{amsmath}
\usepackage{braket}
\usepackage[margin=0.7in]{geometry}
\usepackage{hyperref}
\usepackage[version=4]{mhchem}
% User-Defined Commands
\newcommand{\be}{\begin{equation}}
\newcommand{\ee}{\end{equation}}
\newcommand{\benum}{\begin{enumerate}}
\newcommand{\eenum}{\end{enumerate}}
\newcommand{\pd}{\partial}
% Title Information
\title{Chem132A: Lecture 1}
\author{Shane Flynn (swflynn@uci.edu), Moises Romero (moiseser@uci.edu}

\date{1/8/18}

\begin{document}
\maketitle

\section*{Course Overview}
The course will be broken into two general topics.
The first being the principles of Qantum Mechanics, and t he second being special topics from the literature. 
Students are encouraged to suggest  literature topics to be covered in the second half of the course. 

\subsubsection*{Logistics}
Ther eis no TA for the course, therefore homework willbe assigned and solutions willl be provided, but only the Midterm and Final examinations will be graded.
There are discussion sections for the course (Tuesday and Friday) however, these will only be used to provide make-up lectures. 

\section*{Chapter 5; The Harmonic Oscillator}
We begin the course by discussing Ch.5 in Tannoudji (P.480), the Harmonic Oscillator. 
The HO is a useful model throughout physics because it can be solved analytically (is some cases), and provides an intuition for methods and techniques in Quantum Mechanics. 
Some common problems modeled by the HO are the study of vibrations of atoms about their equiibriun positions, and the oscillations of atoms in a crystalline lattice (photons).
An important example is the electromagnetic field, there  exists an  infinite number of possible stationary waves within a cavity (normal modes of the cavity).
The electromagnetic field can be expanded in these modes and shown to have coefficients obeying differential equations identical to the HO. 
Meaning the electric field is formallye quivalent to a set of indepedent harmonic oscillators. 

The HO essentially assumes we are near a minimum and computes a truncated Taylor Expansion for the Potential Energy (V) around a minimum x$_0$
\be
V(x - x_0) = V(x_0)  + (x-x_0) \left[\frac{dV(x)}{dx}\right]_{x_0} + \frac{1}{2}(x-x_0)^2  \left[\frac{d^2V(x)}{dx^2}\right]_{x_0} + \frac{1}{6} (x-x_0)^3  \left[\frac{d^3V(x)}{dx^3}\right]_{x_0} + \cdots
\ee
The first term in the expansion is a constant and can usually be ignored (we can always re-define the Zero-potential to make this constant 0).
The first derivitive is zero by definition of being in a minimum.
Truncating this expression to second order produces the HO Potential.
\be
\begin{split}
    V(x-x_0) &= V(x_0) +  \frac{1}{2}k(x-x_0)^2\\
    k &\equiv \left[\frac{d^2V(x)}{dx^2}\right]_{x_0}
\end{split}
\ee

Therefore the model replaces the Potential Energy by a porabola, a good approximation near the minimum, and not very good higher along the surface.
In the language of chemistry it can represent lower leveled quantum states, but is inconsistent with higher excitations. 
These higher states are by definition weaker, and therefore can be treated by techniques like Pertebation Theory (which will be covered later in the course). 

Any bound system can be represented by a HO, and it can be used toanalyze the many-body problem (many bodied systems).
Consider a colllection of non-interacting particles (Bosons).
How many atoms can be in an energy level?
Each particle will contribute the characteristic $\hbar \omega$.
Although in this example we are talking about the energy of the particles (and teh energy within each state) this is the same function  form as the HO (which has energy gaps seperated by $\hbar \omega$.
We can therefore treat a many-body problem such as a collection of non-interacting Bosons as a collection of harmonic oscillators. 

\section{Classical Harmonic Oscillator}
In Classical Mechanics the motion of a particle 
Consider the potential energy function for a particle of a mass(m) moving in a potential only dependent on position (x). Where k is the spring constant. 
%\be
%V(x) = \frac{1}{2}kx^2
%\ee

We can determine an expression for Force using : 
\be
F_x=-\frac{\partial V}{\partial x} = -kx
\ee
We can see that the H.O. is a restoring force, and thus the the particle is attraced to x=0 [minimum of potential function V(x)]

A particles motion around x=0 is a sinusoidal of an angular frequency ($\omega$) 
\be
\omega=\sqrt{\frac{k}{m}}
\ee

We can find a mathematical expression for k using the Force expression : 
\be
k=\frac{\partial^2V}{\partial^2x}
\ee

Using Newtons Second Law we can write an equation of motion for the H.O.
\be
F=ma=m\frac{d^2x}{dt^2}=\frac{-dV}{dx}=-kx
\ee
Solving this differntial equation gives the general solution to describe a H.O. motion : 
\be
x=x_Mcos(\omega t - \phi)
\ee
Where $x_M$ and $\phi$ are constants and values are determined by initial conditions of the H.O. 

%This is missing E=T+V and then subsitiing the equation of motion in to prove it is time indepent etc to show where the Hamoltonian is derived from in QM, and then taylor expansion section from the book - Moises









\section{Properties of QM Hamiltonian}
In Quantum mechanics position and momentum are described by their respective operators X and P. 
Which when taking the commutator yieled the following relationsip : 
\be
[X,P]=i\hbar
\ee
\be
[P,X]=-i\hbar
\ee
%needs derivation to be added / explantion - Moises

The Hamiltonian for the quantum Harmonic Oscillator is then taken from the classical representation but taking x and replacing it with the operator X.
\be
H=\frac{p^2}{2m} + \frac{1}{2}m\omega^2X^2
\ee
%The notes has some symmetry operator which i dont understand at all the book has similar notes on it in a small paragraph - Moises
\section{Eigenvalues of the Hamiltonian}

\subsection{$\hat{X}$ and $\hat{P}$ Operators}

X and P have dimensions of legnth and momentum we want to define two new dimensionless operators. We will use S.I. units for dimensional analysis, and recall that angular frequency $\omega$  has units of inverse time,$\hbar$=Js and that a Joule is J=kg*m$^2$s$^{-2}$

For X : 
\be
\hat{X}=\sqrt{\frac{m\omega}{\hbar}}X=\left(\frac{kgs^{-1}}{Js})\right)^{\frac{1}{2}}m = \left(\frac{kg}{kgm^{2}s^{-2}s^{2}}\right)^{\frac{1}{2}}m = \left(\frac{1}{m^2}\right)^{\frac{1}{2}} m = \frac{1}{m}m = 1
\ee
For P : 
\be
\hat{P} = \frac{1}{\sqrt{m\hbar \omega}} P = \frac{1}{\sqrt{kg *J*s*s^{-1}}} P = \frac{1}{\sqrt{kg*kg*m^2s^{-2}}} P = \frac{1}{\sqrt{\frac{kg^2m^2}{s^2}}} P = \frac{1}{\frac{kg*m}{s}}*kg*m*s^{-1} = 1
\ee

Thus we see that $\hat{X}$ and $\hat{P}$ are dimensionless

We now want to see the commutator relationship of our new operators. 
\be
[\hat{X},\hat{P}] = \hat{X}\hat{P} - \hat{P}\hat{X}
\ee


\end{document}

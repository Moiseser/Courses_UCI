% Comments are given  by the % charactee
\documentclass{article}
% Packages
\usepackage[utf8]{inputenc}
\usepackage{graphicx}
\usepackage{amsmath}
\usepackage{braket}
\usepackage[margin=0.7in]{geometry}
\usepackage[version=4]{mhchem}
% User-Defined Commands
\newcommand{\be}{\begin{equation}}
\newcommand{\ee}{\end{equation}}
\newcommand{\benum}{\begin{enumerate}}
\newcommand{\eenum}{\end{enumerate}}
\newcommand{\pd}{\partial}
\newcommand{\dg}{\dagger}
% Title Information
\title{Chem232A: Lecture 1}

\date{1/8/18}
\author{Moises Romero, Alan Robledo, and Shane Flynn}

\begin{document}
\maketitle

\section*{Course Overview}
The course will be broken up into two general topics.
The first being the principles of Quantum Mechanics, and the second being special topics from the literature.
Students are encouraged to suggest literature topics to be covered in the second half of the course.

\subsubsection*{Logistics}
There is no TA for the course. Therefore, homework will be assigned and solutions will be provided, but only the Midterm and Final examinations will be graded.
There are discussion sections for the course (Tuesday and Friday). However, these will only be used to provide make-up lectures.

\section*{Chapter 5; The Harmonic Oscillator}
We begin the course by discussing Ch.5 in Tannoudji (P.480), the Harmonic Oscillator.
The HO is a useful model throughout physics because it can be solved analytically (in some cases), and provides an intuition for methods and techniques in Quantum Mechanics.
Some common problems modeled by the HO are the study of vibrations of atoms about their equilibrium positions, and the oscillations of atoms in a crystalline lattice (phonons).
An important example is the electromagnetic field, there  exists an  infinite number of possible stationary waves within a cavity (normal modes of the cavity).
The electromagnetic field can be expanded in these modes and shown to have coefficients obeying differential equations identical to the HO.
Meaning the electric field is formally equivalent to a set of independent harmonic oscillators.

\subsection*{The HO as a Taylor Expansion}
The HO essentially assumes we are near a minimum and computes a truncated Taylor Expansion for the Potential Energy (V) around the minimum x$_0$
\be
V(x - x_0) = V(x_0)  + (x-x_0) \left[\frac{dV(x)}{dx}\right]_{x_0} + \frac{1}{2}(x-x_0)^2  \left[\frac{d^2V(x)}{dx^2}\right]_{x_0} + \frac{1}{6} (x-x_0)^3  \left[\frac{d^3V(x)}{dx^3}\right]_{x_0} + \cdots
\ee
The first term in the expansion is a constant and can usually be ignored (we can always re-define the Zero-potential to make this constant 0).
The first derivative is zero by definition of being at a minimum.
Truncating this expression to second order produces the HO Potential directly from the Taylor Expansion about a minimum.
\be
\begin{split}
    V(x-x_0) &= V(x_0) +  \frac{1}{2}k(x-x_0)^2\\
    k &\equiv \left[\frac{d^2V(x)}{dx^2}\right]_{x_0}
\end{split}
\ee

Therefore the model replaces the Potential Energy by a parabola, a good approximation near the minimum, and not very good higher along the surface.
In the language of chemistry it can represent lower levele quantum states, but is inconsistent with higher excitations.
These higher states are by definition weaker, and therefore can be treated by techniques like Perturbation Theory (which will be covered later in the course).

Any bound system can be represented by a HO, and it can be used to analyze the many-body problem (many bodied systems).
Consider a collection of non-interacting particles (Bosons).
How many atoms can be in an energy level?
Each particle will contribute the characteristic $\hbar \omega$.
Although in this example we are talking about the energy of the particles (and the energy within each state) this is the same function  form as the HO (which has energy gaps separated by $\hbar \omega$.
We can therefore treat a many-body problem such as a collection of non-interacting Bosons as a collection of harmonic oscillators.

\subsection*{Classical Harmonic Oscillator Approach}
In classical mechanics we can approach the HO problem from a different angle. 
A fairly simple way of seeing the parabolic nature of the potential energy of a harmonic oscillator is by introducing the widely used scenario of an object with mass m that is attached to a massless spring and is moving periodically. 
If $\vec{F}$ is the only force involved in extending or compressing the spring, the system is called a simple Harmonic Oscillator (HO) and we say that the system obeys Hooke's law. 
Therefore, the force acting on the mass from the spring is the restoring force and has the form $\vec{F} = - k \vec{r}$.
From Newtons Second Law, we can now write an equation of motion for the HO
\be
\vec{F}=m\vec{a}=m\frac{d^2\vec{r}}{dt^2}=-k\vec{r} .
\ee
This gives us a second order linear differential equation with constant coefficients. 
If we consider the one dimensional case where the force acts only in the x-direction, we can reduce the number of differential equations down to one. 
Rearranging the terms and setting the equation equal to 0 gives us
\be
\frac{d^2x}{dt^2} + \omega^2 x = 0.
\ee
where $\omega^2 = \frac{k}{m}$. $\omega$ in this case can be thought of as the angular frequency of the HO. Solving this differential equation gives the general solution that
describes the motion of the HO
%===========================================================================%
%================Insert Math for the General Solution=======================%
%===========================================================================%
\be
x=x_Mcos(\omega t - \phi)
\ee
where $x_M$ and $\phi$ are constants and values that are determined by the initial conditions of the HO.

In order to find the potential energy stored in the HO, all we need to do is find out the amount of work done on the system as a result of a perturbation (i.e. stretching or
compressing the spring attached to the mass). Going back to Hooke's law, the force needed to stretch or compress the spring is $\vec{F} = k \vec{x}$ (remember that we are
only considering a force in the x-direction). Do not get this confused with the restoring force that the spring exerts on the mass after the perturbation which is $\vec{F} = - k \vec{x}$.
Since we have the appropriate force and are able to keep track of the object's displacement over time, we can calculate work as
\be
\begin{split}
W &= \int_{0}^{x} \vec{F}_x \cdot d\vec{x} \\
&= \int_{0}^{x} k x dx \\
&= \frac{1}{2}k (\Delta x)^2 .
\end{split}
\ee
We calculated the work to eventually lead into saying that the change in elastic potential energy of the system is equal to the amount of work done on the system $\Delta V = W = \frac{1}{2}k (\Delta x)^2$. Expanding the deltas on both sides gives us the stored potential energy in the HO
\be
V(x) = \frac{1}{2}k x^2.
\ee
By doing some differentiation instead of integration, we can see that the force constant k can be expressed as
\be
k \equiv \frac{d^2 V(x)}{d x^2}.
\ee
Setting the equilibrium position of the mass m to be at x = 0 means that the minimum of the potential function V(x) occurs at the equilibrium position. 
Relating this to chemistry and the fact that the harmonic oscillator is a useful model, if you had a diatomic molecule and considered one of the atoms, say the left atom, to be grounded and you pulled the right atom away from the left, you would be moving the atom away from the equilibrium position and thus stretching the bond. 
In terms of potential energy, the potential would be increasing parabolically. 
If you let go of the right atom, it would move to its equilibrium position where the potential energy of the molecule is at a minimum and where the molecule can be thought of as being in the ground state.

%============================================================================%
%============This needs tobe incorporated into the ocument so it flows=======%%
%============================================================================%

We calculated the work to eventually lead into saying that the change in elastic potential energy of the system is equal to the amount of work done on the system $\Delta V = W = \frac{1}{2}k (\Delta x)^2$. Expanding the deltas on both sides gives us the stored potential energy in the HO
\be
V(x) = \frac{1}{2}k x^2.
\ee
By doing some differentiation instead of integration, we can see that the force constant k can be expressed as
\be
k \equiv \frac{d^2 V(x)}{d x^2}.
\ee
Setting the equilibrium position of the mass m to be at x = 0 means that the minimum of the potential function V(x) occurs at the equilibrium position. Relating this to chemistry
and the fact that the harmonic oscillator is a useful model, if you had a diatomic molecule and considered one of the atoms, say the left atom, to be grounded and you pulled the
right atom away from the left, you would be moving the atom away from the equilibrium position and thus stretching the bond. In terms of potential energy, the potential would be
increasing parabolically. If you let go of the right atom, it would move to its equilibrium position where the potential energy of the molecule is at a minimum and where the
molecule can be thought of as being in the ground state.

Since we are simply dealing with one object and the spring is massless, we can just say that the kinetic energy is
\be
T = \frac{1}{2}m (\frac{\partial x}{\partial t})^2 = \frac{1}{2}m v^2.
\ee
Likewise, we can express the kinetic energy of the mass in terms of its momentum $p$
\be
T = \frac{p^2}{2 m} = \frac{1}{2} m v^2
\ee
where $p = mv$. By combining the kinetic and potential energies, we get the total energy energy of the system
\be
E = T + V = \frac{1}{2}mv^2 + \frac{1}{2}kx^2
\ee
but it can also be expressed with $\omega$ and $p$
\be
E = T + V = \frac{p^2}{2 m} + \frac{1}{2}m  \omega^2 x^2 .
\ee
%This is missing E=T+V and then subsitiing the equation of motion in to prove it is time indepent etc to show where the Hamoltonian is derived from in QM, and then taylor expansion section from the book - Moises







A final comment, at higher Energies the partical will move in periodic, but not sinusoidal motion.
If you expand the function info a Fourier series to get the position of the particle you find several frequencies or integer multiples of the lowest frequency.
This is an anharmonic oscillator and the period of this motion is more complicated. 

\section*{Quantum Mechanics Harmonic Oscillator}
From undergraduate QM we know we want to solve the Schrodinger Equaiton for our sustems. 
In the context of the Quantum HO we construct the Hamiltonian Operator as
\be
\begin{split}
    \hat{H}\Psi &= E\Psi\\
    \hat{H} &= \frac{\hat{P}^2}{2m} + \frac{1}{2}m\omega^2 \hat{X}^2
\end{split}
\ee
If we assume a conservative system $\hat{H}$ is a time-indepedent operator, and assuming a 1D (x) system we can express the HO written in the x  representation as
\be
\begin{split}
    \hat{H}\ket{\phi} &= E\ket{\phi}\\
    \left[-\frac{\hbar^2}{2m} \frac{d^2}{dx^2} + \frac{1}{2}m\omega^2x^2\right]\phi(x) &= E\phi(x)
\end{split}
\ee
From this equation the undergraduate course explores various methods of finding interesting properties (eigenvalues, energy, etc). 
\be
\braket{A} = \int \phi^*\hat{A}\phi
\ee

We will take a new  approach in this class, by introducing new operators. 
The \textbf{Creation} and \textbf{Annilation} operators can be used to solve all of the properties associated with $\hat{H}$ without any differential equaitons!
We will calculate these quantities in a more elegant manner, bypassing the wavefunction all together!
In the case of the HO we can solve the problem exactly so either method is sufficient.
But in reality more problems are not exact and the wavefunction approach will nto be useful.
So we need to develop a method of applying operator algebra to bypass the wavefunction. 

\subsection*{Basic QM Review}
\subsubsection*{Commutators}
It will be essential to understand commutation relationships for our operator algebra. 
In,QM  position and momentum are described using their operators $\hat{x}$ and $\hat{p}$.
We recall tht these two opertors do not commute in either orientation.
\be
\begin{split}
[\hat{x},\hat{p_x}] &= i \hbar \\
[\hat{p_x},\hat{x}] &= - i \hbar
\end{split}
\ee
To derive this result introduce a generic function f(x), and apply it to the commutation relationship.
Recall in general that the commutator essentially measures ``how much" two operators commute by taking the difference in orientation of the two operators.
\be
[\hat{x},\hat{p_x}] = \hat{x}\hat{p_x} - \hat{p_x}\hat{x}
\ee
Applying our funciton f(x) to the definition of the commutator yields:
\be
[\hat{x},\hat{p_x}]f(x) = (\hat{x} \hat{p_x} - \hat{p_x} \hat{x})f(x) = \hat{x} \hat{p_x}f(x) - \hat{p_x} \hat{x}f(x)
\ee
Taking the definition of each operator, $\hat{x} = x $ and $\hat{p_x} = - i \hbar \frac{d}{dx}$, we can make the appropriate substitutions.
\be
\hat{x} \hat{p_x}f(x) - \hat{p_x} \hat{x}f(x) = - x i \hbar \frac{df(x)}{dx} - (- i \hbar \frac{d(xf(x))}{dx})
\ee
Using the product rule and cancelling out terms, we get
\be
\begin{split}
    [\hat{x},\hat{p_x}]f(x) &= - x i \hbar \frac{df}(x){dx} + i \hbar(f(x) + x \frac{df}(x){dx}) \\
    &= - x i \hbar \frac{df(x)}{dx} + i \hbar f(x) + x i \hbar \frac{df(x)}{dx} \\
    &= i \hbar f(x)
\end{split}
\ee
Therefore the generic commutation (which is not depedent upon our generic function f(x) is)
\be
[\hat{x},\hat{p_x}]= i \hbar
\ee
A major reason QM is interesting is because these two operators do not commute. 
This commutator yielding a nonzero operator is directly connected to the uncertainty principle!
\be
\sigma_x \sigma_p \geq \frac{\hbar}{2}.
\ee
If a commutator yields the zero operator $\hat{0}$, we say that the two operators commute.
Physically speaking the observables involved in the commutator can be measured simultaneously to arbitrary precision.
We have just shown that the position and momentum (in the same dimension) cannot be measured simultaneously to arbitrary precision. 
By recognizing that $\sigma_x$ and $\sigma_p$ are standard deviations of the position and momentum, we can say that the minimum amount of uncertainty in the simultaneous measurement is equal to $\frac{\hbar}{2}$.
%============================================================================%
% Do the other derivation out giving the reverse result
%============================================================================%

\subsubsection*{Eigenvalues Are Positive (M3 Complement)}
We can show the energies ofthe HO are positive.
This is a simple conservative system, therefore the total energy is given by the sum of the Kinetic and Potential Energy respectively (consider bound stataes for htis problem)
\be
\hat{E} = \braket{\hat{T}} + \braket{\hat{V}}
\ee
We know the Kinetic Energy Operator is given by
\be
\braket{T} = -\frac{\hbar^2}{2m}\int_{-\infty}^\infty dx \phi^*(x) \frac{d^2}{dx^2} \phi(x) = \frac{\hbar^2}{2m} \int_{-\infty}^\infty dx \left|\frac{d}{dx}\phi(x)\right|^2 \geq 0
\ee
Where we have performed integration by parts, and used the fact that teh wavefunction goes to 0 at an infinite distance, showing this value is non-negative.

We can now look at the Potential Energy
\be
\braket{V} = \int_{-\infty}^\infty dx V(x) |\phi|^2
\ee
Defining the minimum of the potential to be -V$_0$, than we know that all values must be larger than -V$_0$, generating the inequality:
\be
\braket{V} \geq \int_{-\infty}^\infty dx (-V_0) |\phi|^2 = -V_0 \int_{-\infty}^\infty dx |\phi|^2 = -V_0
\ee
Where the equality comes from the normalization. 
Having shown the Kinetic Energy to be non-negative we have shown that
\be
E = \braket{T} + \braket{V} > \braket{V} \geq -V_0
\ee

\subsubsection*{Parity in the Harmonic Oscillator}
The parity of a function asks what happens when you take the inverse of the inputs.
In general the Potential Energy of a generic function does not need to be symmetirc wrt parity ($\pi$), however, in many cases it is. 
\be
\begin{split}
    \pi V(x) &= V(x) \quad \quad \text{symetric}\\
    \pi V(x) &= -V(x) \quad \quad \text{anti-symetric}\\
\end{split}
\ee
Clearly the HO Potential Energy  must be symmetric wrt parity (it goes as V(x) $\approx$ $x^2$) for this specific system. 
This is not necessiarly true for the wavefunction however, it can be either symmetric or anti-symmetric.
For example the HO wavefunction alternates symmetry (check teh Hermite Polynomials). 

\section*{Eigenvalues of the Hamiltonian}
We will now prove the eigenvalues of teh HO are discrete without using the Schrodinger Equation. 
We know in general the Eigenvalue problem is interested in teh following 
\be
H\ket{\phi} = E\ket{\phi}
\ee
We will start by introducing some notation. 
The position and momentum operators definitions must have dimensions (of positon and momentum...).
We can define some dimensionless units by noting that $\omega$ by construction has units of inverse time and $\hbar$ is na action (energy*time).
If we make the following definitions, our units will be dimensionless in these new coordinates
\be
\hat{X} = \sqrt{\frac{m\omega}{\hbar}}X, \qquad \hat{P} = \frac{1}{\sqrt{m\hbar \omega}} P
\ee

\subsubsection*{Dimensional Analysis}
For X :
\be
\hat{X}=\sqrt{\frac{m\omega}{\hbar}}X=\left(\frac{kgs^{-1}}{Js})\right)^{\frac{1}{2}}m = \left(\frac{kg}{kgm^{2}s^{-2}s^{2}}\right)^{\frac{1}{2}}m = \left(\frac{1}{m^2}\right)^{\frac{1}{2}} m = \frac{1}{m}m = 1
\ee
For P :
\be
\hat{P} = \frac{1}{\sqrt{m\hbar \omega}} P = \frac{1}{\sqrt{kg *J*s*s^{-1}}} P = \frac{1}{\sqrt{kg*kg*m^2s^{-2}}} P = \frac{1}{\sqrt{\frac{kg^2m^2}{s^2}}} P = \frac{1}{\frac{kg*m}{s}}*kg*m*s^{-1} = 1
\ee
Thus we see that $\hat{X}$ and $\hat{P}$ are dimensionless

We can now evaluate the commutator for these new coordinates in a similar manner to before. 
\be
[\hat{X},\hat{P}] = \hat{X}\hat{P} - \hat{P}\hat{X} = i
\ee
\be
[\hat{P},\hat{X}] = \hat{P}\hat{X} - \hat{X}\hat{P} = \frac{1}{i}
\ee
\subsubsection*{Proof}
%=============================================================================%
% Someone else can do this algebra I don't feel like it 
% Please do both!!!!
%=============================================================================%

\subsection*{Hamiltonian}
Using our new unitless definition the Hamiltonian can be written as 
\be \label{eq:Hamiltonian}
\begin{split}
    H &= \hbar \omega \hat{H}\\
    \hat{H} &\equiv \frac{1}{2}\left(\hat{X}^2 + \hat{P}^2\right)
\end{split}
\ee

\subsubsection*{Algebra}
\be
\begin{split}
    \hat{x}^2= \frac{m\omega}{\hbar}x^2 \qquad & \qquad \hat{P}^2 = \frac{1}{m\hbar\omega}P^2 \\
    H &= \hbar\omega\hat{H} = \hbar\omega \left[\frac{1}{2} \left(\frac{m\omega}{\hbar} X^2 + \frac{1}{m\hbar\omega}P^2\right)\right]\\
    &= \frac{m\omega^2}{2} X^2 + \frac{1}{2m}P^2
\end{split}
\ee

We are now going to solve the related eigenvalue equation
\be
\hat{H}\ket{\phi_\nu^i} = \epsilon_\nu \ket{\phi_\ni^i}
\ee
Where the operator $\hat{H}$ and the eigenvalues $\epsilon$ are dimensionless. 
The index $\nu$ can be either continuous or discrete, and the index i allows us to distinguish between teh various possible orthogonal eigenvecctors associated with the same eigenvalue $\epsilon_i$. 

Because the $\hat{X}$ and $\hat{P}$ operators do  not commute:
\be
\hat{X}^2 + \hat{P}^2 \neq (\hat{X} - i\hat{P})(\hat{X} + i\hat{P})
\ee
We shall show that introducing twonew  operators proportional to these  factor swill helpo simplify the eigen problem considerably. 
\be
a \equiv \frac{1}{\sqrt{2}} (\hat{X} + i\hat{P}) \qquad a^\dg \equiv \frac{1}{\sqrt{2}} (\hat{X} - i\hat{P})
\ee
If  you invert these formulas than you find
\be
\hat{X} = \frac{1}{\sqrt{2}} (a^\dg + a) \qquad \hat{P} = \frac{1}{\sqrt{2}} (a^\dg - a)
\ee
Note: these two operators are not  Hermitian (X and P are hermitian, a and a$^\dg$ have a factor of i and are therefore not Hermitian). 

The commutator for our new operators is given by
\be
[a,a^\dg] = 1
\ee
\subsubsection*{Proof}
\be
 \begin{split}
     [a,a^\dg] &= aa^\dg - a^\dg a =\\
     \frac{1}{\sqrt{2}} \left(\hat{X} + i\hat{P}\right) \frac{1}{\sqrt{2}} \left(\hat{X} - i\hat{P}\right) &- \frac{1}{\sqrt{2}} \left(\hat{X} - i\hat{P}\right)\frac{1}{\sqrt{2}} \left(\hat{X} + i\hat{P}\right)\\
   \frac{1}{2} \bigg\{ \left(\hat{X} + i\hat{P}\right) \left(\hat{X} - i\hat{P}\right) &- \left(\hat{X} - i\hat{P}\right) \left(\hat{X} + i\hat{P}\right) \bigg\}\\
      \frac{1}{2} \bigg\{\hat{X}^2 - \hat{X}i\hat{P} + i\hat{P}\hat{X} + \hat{P}^2  &- \left(\hat{X}^2 + \hat{X}i\hat{P} - i\hat{P}\hat{X} + \hat{P}^2\right)  \bigg\}\\
        \frac{1}{2} \bigg\{\hat{X}^2 + \hat{P}^2 + i\left(\hat{P}\hat{X} - \hat{X}\hat{P}\right)   &- \left(\hat{X}^2 + \hat{P}^2 + i\left(\hat{X}\hat{P} - \hat{P}\hat{X}\right)\right) \bigg\}\\
        \frac{i}{2} \bigg\{ \left[\hat{P},
        \hat{X}\right] &- \left[\hat{X},
     \hat{P}\right] \bigg\}\\
     \frac{i}{2} \bigg\{ \frac{1}{i} &- i] \bigg\}\\
     [a,a^\dg] &= 1
 \end{split}
 \ee

If we want the reverse order it should give
\be
[a^\dg, a] = -1
\ee
\subsubsection*{Proof}
%====================================================================================%
% I am guessing but pretty sure, someone should chekc/do this
%====================================================================================%


It turns out we can write a very convenient form of the Hamiltonian opertor in terms of our new operators.
Consider the following
\be
\begin{split}
a^\dg a &= \left(\hat{X} - i\hat{P}\right) \left(\hat{X} + i\hat{P}\right)\\
&= \frac{1}{2}\left(\hat{X}^2 + \hat{P}^2 + i\left\{ \hat{X}\hat{P} - \hat{P}\hat{X}\right\}\right)\\
a^\dg a &=\frac{1}{2}\left(\hat{X}^2 + \hat{P}^2 - 1\right)
\end{split}
\ee
Now compare this to our definition of the Hamiltonian Operator in the normalizeed units (eq. \ref{eq:Hamiltonian}) we immediately see that we can write
\be
\hat{H} = a^\dg a + \frac{1}{2}
\ee
We can also write the expression as 
\be
\hat{H} = aa^\dg - \frac{1}{2}
\ee
%===========================================================================================%
% Someone show this transformation too
%===========================================================================================%
\subsection*{Number Operator}
For convenience (as we will see later) we can ndefine a new operator called teh \textbf{Number Operator}
It turns out this operator is related to the number of excitations occuring in the oscillator. 
\be
N \equiv a^\dg a
\ee
This operator is Hermitian (unlike a or a$^\dg$.
%===========================================================================================%
% Someone show this, I don't honestly remember how
%===========================================================================================%
Given our insight, this operator counts the number of excitations that occur in the oscillator.
It is therefore a physical quantity, and muyst be a Hermitian operator!	

Taking our definition of the Hamimltonian Operator we can also express the relationship as
\be
\hat{H} = N + \frac{1}{2}
\ee
Which implies the eigenvectors of $\hat{H}$ are also eigenvectors of N. 
\be
\begin{split}
N\ket{\phi_\nu} &= \nu \ket{\phi_\nu}\\
H\ket{\phi_\nu} &= (\nu+\frac{1}{2})\hbar\omega \ket{\phi_\nu}\\
\end{split}
\ee
Where we are using the unscaled Hamiltonian operator, and the factor of $\frac{1}{2}$ difference from teh Number Operator.

We can quickly calculate the commutators for N and a
\be
\begin{split}
[N,a] &= [a^\dg a, a] = a^\dg[a,a] + [a^\dg,a]a = -a\\
[N,a^\dg] &= [a^\dg a, a^\dg] = a^\dg[a,a^\dg] + [a^\dg,a^\dg]a = a^\dg\\
\end{split}
\ee

\subsection*{Creation/Annilation}
The a, a$^\dg$ operators are known as the \textbf{Annilation} and \textbf{Creation} operators (which we will show later in the course).
Consider an arbitrary eigenvector $\ket{\phi_\nu}$ of N.
The square modulus must be either 0 or larger than 0 (the length of the modulus cannot be negative).
\be
||a\ket{\phi_\nu}||^2 = \braket{\phi_\nu|a^\dg a| \phi_\nu} \geq 0
\ee

If $\nu$ = 0, than a$\ket{\phi_\nu}$ =0.
If $\nu > 0$ than a$\ket{\phi_\nu}$ is the eigenfunction of N with an eigenvalue lowered by 1. 

Consider a generic vector, multiplyign each sife by a$^\dg$ yields the number operator. 
\be
\begin{split}
    a\ket{\phi} &= 0 \\
    a^\dg a \ket{\phi} &= N\ket{\phi} = 0
\end{split}
\ee
So any vector satisfying our assumption of $a\ket{\phi} = 0$ is an eigenvector of N with an eigenvalue of 0. 

Cnsider now a non-0 positive value of $\nu$. 
From the N,a commutator we can write the following
\be
\begin{split}
    [N,a]\ket{\phi_\nu} &= -a\ket{\phi}\\
    Na\ket{\phi_\nu} - aN\ket{\phi_\nu} &= -a\ket{\phi_\nu}\\
    Na\ket{\phi_\nu} &= aN\ket{\phi_\nu} -a\ket{\phi_\nu}\\
    Na\ket{\phi_\nu} &= a\nu\ket{\phi_\nu} -a\ket{\phi_\nu}
\end{split}
\ee
This tell sus we can write 
\be
Na \ket{\phi_\nu} = (\nu-1)a\ket{\phi_\nu}
\ee
Which shows acting a to the right will lower the state by 1.
We can therefore use this relationship to compute all of the lower states. 

We can also work this problem in the other direction using a$^\dg$.
\be
\begin{split}
    [N,a^\dg]\ket{\phi_\nu} &= a^\dg\ket{\phi}\\
    Na^\dg\ket{\phi_\nu} - a^\dg N\ket{\phi_\nu} &= a^\dg\ket{\phi_\nu}\\
    Na^\dg\ket{\phi_\nu} &= a^\dg N\ket{\phi_\nu} + a^\dg\ket{\phi_\nu}\\
    Na^\dg\ket{\phi_\nu} &= a^\dg \nu\ket{\phi_\nu} + a^\dg\ket{\phi_\nu}\\
    Na^\dg \ket{\phi_\nu} &= (\nu+1)a^\dg\ket{\phi_\nu}
\end{split}
\ee
Showing the a$^\dg$ operator can be used to find higher exciattions. 

%====================================================================================%
% The book goes through some logic arguing this, someone else can walk through it if they want
%====================================================================================%
The eigenvalues must be integers, if non-integer than you could construct an eigenvalue less than 0 which is unphysical. 

We need to reaalize, we have drived all of the eigenvalues and eigenvectors with an arbitrary wavefunction, no evaluating, just algebra!
In general you cannot always solve for the wavefunction, these algebraic methods are much more reliable for harder problems. 





\end{document}

\documentclass{article}
% Packages
\usepackage[utf8]{inputenc}
\usepackage{graphicx}
\usepackage{amsmath}
\usepackage{braket}
\usepackage[margin=0.7in]{geometry}
\usepackage{hyperref}
\usepackage[version=4]{mhchem}
% User-Defined Commands
\newcommand{\be}{\begin{equation}}
\newcommand{\ee}{\end{equation}}
\newcommand{\benum}{\begin{enumerate}}
\newcommand{\eenum}{\end{enumerate}}
\newcommand{\pd}{\partial}
\newcommand{\dg}{\dagger}
% Title Information
\title{Chem231B: Lecture 2}
\author{Shane Flynn}
\date{1/9/18}

\begin{document}
\maketitle

%=================================================================================================%
$\approx$ P496 Volume 1 Cohen-Tannoudji
%======================Add some page numbers to help follow along in the future===================%
%=================================================================================================%
\section*{Operator Algebra}
A cornerstone of this course will be the use of operators to solve Quantum Mechanics Problems. 
By using the operator formulation of QM we can completly ignore the wavefunction and instead compute observables from just the algebraic manipulations of opertors.
This means we can ignore solving the various integrals and differential equations commonly found in the wave function formulation. 

\section*{The Ground State}
We can consider the ground state energy of the HO, which has an eigenvector satisfying the following equation
\be
a\ket{\phi_0} = 0
\ee
This statement is really a differential equation, remember what the operators represent
\be
\left\{\frac{1}{\sqrt{2}} \sqrt{\frac{m\omega}{\hbar}}x + \frac{i}{\sqrt{m\hbar\omega}}p\right\}\phi_0 = 0
\ee
In the position representation  we have
%============================================================================================%
%Someone should show why this is true derivations
%============================================================================================%
\be
\left(\frac{m\omega}{\hbar}x + \frac{d}{dx}\right)\phi_0(x) = 0
\ee
%============================================================================================%
%Someone should prove the general solution
%============================================================================================%
The general solution to this differential equation is of the form 
\be
\phi(x) = ce^{-\frac{m\omega x^2}{2\hbar}}
\ee
This solution (which takes some effort) is just for the ground state.
All of the solution are actually proportional, therefore only 1 ket exists to describe the ground state (it is degenerate) giving an energy of
\be
E_0 = \frac{\hbar\omega}{2}
\ee

\subsection*{First Excited State}
We want more, how do we approach the excited states?
To do this we will utilize the creation and annilation operators.
It is important to realize that a and a$^\dg$ do NOT conserve normaliztion, as we will see below.

We can show that all of the states are also non-degenerate.
Suppose we have a single vector satisfying 
\be
N\ket{\phi_n} = n\ket{\phi_n}
\ee
Likewise we have an eigenvector associated with the n+1 eigenvalue.
\be
N\ket{\phi_{n+1}} = (n+1)\ket{\phi_{n+1}}
\ee
We also know from Lecture 1 that our annilation operator can be used to write
\be
a\ket{\phi_{n+1}} = c\ket{\phi_n}
\ee
If we simply stick the a$^\dg$ operator in this expression we have a nice simplification
\be
\begin{split}
    a^\dg a\ket{\phi_{n+1}} &= a^\dg c\ket{\phi_n}\\
    N\ket{\phi_{n+1}} &= a^\dg c\ket{\phi_n}\\
    (n+1)\ket{\phi_{n+1}} &= a^\dg c\ket{\phi_n}\\
    \ket{\phi_{n+1}} &= \frac{c}{(n+1)}a^\dg\ket{\phi_n}\\
\end{split}
\ee
This result shows all n+1 vectors are proportional to a$^\dg \ket{\phi_n}$ and therefore proportional to eachother and the eigenvalues are not degenerate. 
\be
\ket{\phi_n} \rightarrow \qquad \qquad E_n = \left(n + \frac{1}{2}\right)\hbar\omega
\ee

Our result also shows that we need to know the pre-factors to get the wavefunctions.
\be
a\ket{\phi_0} = 0, \qquad \ket{\phi_1} = c_1a^\dg\ket{\phi_o} \cdots
\ee

We can find our normalization by taking the scalar product
\be
\begin{split}
    \braket{\phi_1|\phi_1} &= |c_1|^2 \braket{\phi_0|aa^\dg|\phi_1}\\
    &= |c_1|^2 \braket{\phi_0|(a^\dg a+1)|\phi_1}
\end{split}
\ee
The last line follows because of the commutator
\be
[a,a^\dg] = 1 \rightarrow aa^\dg - a^\dg a = 1 \rightarrow  aa^\dg = 1 + a^\dg a
\ee
If we require $\ket{\phi_1}$ to be normalized and have a constant c$_1$ to be real and positive (relative to the phase of $\ket{\phi_0}$.
But $\ket{\phi_0}$ is a normalized eigenstate of N with an eigenvalue of zero as we have shown, therefore
\be
\braket{\phi_1|\phi_1} = |c_1|^2 = 1, \qquad c_1 = 1
\ee
We can always have an arbitrary phase term associate with $\phi$ that is fine, it will jsut make $\phi$ complex, choosing c$_1$ to be 1 makes $\phi$ real. 

\subsection*{Second Excited State}
In the same manner we can construct the next state using our operators, assuming c$_2$ to be real and $\ket{\phi_2}$ tp be normalized. 
\be
\begin{split}
    \ket{\phi_2} &= c_2 a^\dg\ket{\phi_1}\\
    \braket{\phi_2|\phi_2} &= |c_2|^2 \braket{\phi_1|aa^\dg|\phi_1}\\
    &= |c_2|^2 \braket{\phi_1|(a^\dg a + 1)|\phi_1}\\
    &= |c_2|^2 \braket{\phi_1|(N + 1)|\phi_1}\\
    &= |c_2|^2 \braket{\phi_1|(N\phi_1 + \phi_1}\\
    &= |c_2|^2 \braket{\phi_1|\phi_1 + \phi_1}\\
    &= |c_2|^2 2\braket{\phi_1|\phi_1}\\
    &= 2|c_2|^2  = 1
\end{split}
\ee
We have therefore shown (taking the normaliztion into accoutn)
\be
\ket{\phi_2} = \frac{1}{\sqrt{2}}a^\dg\ket{\phi_1} = \frac{1}{\sqrt{2}}(a^\dg)^2\ket{\phi_0}
\ee

\subsection*{General Solution Harmonic Oscillator}
Hopefully now you can realize that we can build all of the wavefuntions by multiplying with a$^\dg$ and findng the appropriate normaliztion. 
The general case works in teh same manner
\be
\begin{split}
    \ket{\phi_n} &= c_n \ket{\phi_{n-1}}\\
    \braket{\phi_n|\phi_n} &= |c_n|^2 \braket{\phi_{n-1}|aa^\dg|\phi_{n-1}}\\
    \braket{\phi_n|\phi_n} &= |c_n|^2 \braket{\phi_{n-1}|a^\dg a + 1|\phi_{n-1}}\\
    \rightarrow c_n &= \frac{1}{\sqrt{n}}
\end{split}
\ee
\be
\ket{\phi_n} = \frac{1}{\sqrt{n}}a^\dg \ket{\phi_{n-1}} = \frac{1}{\sqrt{n!}}(a^\dg)^n \ket{\phi_0}
\ee
Where this last line represents the general solution for the Harmonic Oscillator!

\subsubsection*{Orthonormal and Closure}
Because H is Hermitian the HO kets $\ket{\phi_n}$ are orthogonal.
Each is individually normalized therefore we have an orthonormal set.
\be
\braket{\phi_{n'} | \phi_n} = \delta_{nn'}
\ee
We also know that H is an observable and we therefore have closure over the basis.
\be
\sum_n \ket{\phi_n} \bra{\phi_n} = 1
\ee

\subsection*{New Section}
Any generic operator A (representing a physical observable) ca be expressed in terms of a and a$^\dg$ because X and P are simply linear combinations of a and a$^\dg$.
This essentially means wecan solve any expectation value by using a and a$^\dg$. 

We have already found general expressions relating a and a$^\dg$ to the HO $\ket{\phi_n}$.
These expressions make it very convenient to operate on $\ket{\phi_n}$.
\be
\begin{split}
    a^\dg\ket{\phi_n} &= \sqrt{n+1}\ket{\phi_{n+1}}\\
    a\ket{\phi_n} &= \sqrt{n} \ket{\phi_{n-1}}
\end{split}
\ee
Which makes it clear where the names creation and annilation  operators come from. 
The second relatioship follows from the commutator of $[a,a^\dg]$
\be
a\ket{\phi_n} = a\frac{1}{\sqrt{n}}a^\dg \ket{\phi_{n-1}} = \frac{1}{\sqrt{n}}aa^\dg \ket{\phi_{n-1}} = \frac{1}{\sqrt{n}}(a^\dg a+1) \ket{\phi_{n-1}} = ? = \sqrt{n}\ket{\phi_{n-1}}
\ee
%===============================================================================================%
% Need to fill in the last line of this derivation P.498
%===============================================================================================%
The creation and annilation names are intuitive when acting on the ket, however, acting on teh bra we actually find the inverse. 
\be
\begin{split}
    \bra{\phi_n}a &= \sqrt{n+1}\bra{\phi_{n+1}}\\
    \bra{\phi_n}a^\dg &= \sqrt{n}\bra{\phi_{n-1}}\\
\end{split}
\ee

\subsection*{Matrix Elements}
From our operator definitions we know
\be
\hat{X} = \sqrt{\frac{m\omega}{\hbar}}X, \quad \hat{X} = \frac{1}{\sqrt{2}}\left(a^\dg + a\right)
\ee
Substituting these expressions we can write the following
\be
X\ket{\phi_n} =  \sqrt{\frac{\hbar}{2m\omega}} \left(a^\dg + a\right)\ket{\phi_n}
\ee
And in the analagous manner we can define the momentum operation as 
\be
P\ket{\phi_n} = \sqrt{m\hbar\omega} \frac{i}{\sqrt{2}} \left(a^\dg - a \right) \ket{\phi_n}
\ee

With all of these expressions we can analyze various matrix elements (recall the HO vectors are orthornormal).
\be
\braket{\phi_{n'}|a|\phi_n} =  \braket{\phi_{n'}|\sqrt{n}\phi_{n-1}} = \sqrt{n}\braket{\phi_{n'}|\phi_{n-1}} = \sqrt{n} \delta_{n',n-1}
\ee
In the same manner we can easily write 
\be
\braket{\phi_{n'}|a^\dg|\phi_n} =\sqrt{n+1} \delta_{n',n+1}
\ee
\be
\braket{\phi_{n'}|X|\phi_n} = \sqrt{\frac{\hbar}{2m\omega}}\left[\sqrt{n+1}\delta_{n',n+1} + \sqrt{n} \delta_{n',n-1}\right]
\ee
\be
\braket{\phi_{n'}|P|\phi_n} = i\sqrt{\frac{m\hbar\omega}{2}}\left[\sqrt{n+1}\delta_{n',n+1} - \sqrt{n} \delta_{n',n-1}\right]
\ee

From these matrix elements it becomes clear that only a and a$^\dg$ are Hermitian Congugates and the matrix a$^\dg$a only has non-zero elements for the diagonal which starts at 0 and increases by 1 along the diagonal. 

\subsection*{Wavefunction}
We can now start discussing the wavefunction itself.
We know from before that the groundstate  is proportional to a Gaussian.
As stated before, ew can get all of the expectation values using a and a$\dg$ because they are proportional to linear combinations of X and P.
The groundstate of the HO (well known) can be written explicitly in the $\{\ket{x}\}$ representation as
\be
\phi_0(x) = \braket{x|\phi_0} = \left(\frac{m\omega}{\pi\hbar}\right)^{1/4} \text{exp} \left\{-\frac{m\omega}{2\hbar}x^2\right\}
\ee
%============================================================================================%
% Could derive this if we felt like it
%============================================================================================%
To obtain the $\phi_n(x)$ associated with the other stationary states (eigenstates are stationary, linear combinations are not) we can use our general solution to the HO vectors and the x representation of a$\dg$
\be
a^\dg = \frac{1}{\sqrt{2}} \left(\hat{X} - i\hat{P}\right) = \frac{1}{\sqrt{2}} \left\{ \sqrt{\frac{m\omega}{\hbar}} x - \sqrt{\frac{\hbar}{m\omega}}\frac{d}{dx}\right\}
\ee

Note: we showed all of the wavefunctions are of Gaussian form, x and p will not change the distribution (it will still be gaussian if you multiply by x or take a derivitive wrt x). 
From our general solution to the HO we can get all of the wavefunctions in the x representation. 
\be
\begin{split}
    \phi_n(x) = \braket{x|\phi_n} &= \frac{1}{\sqrt{n}} \braket{x|(a^\dg)^n|\phi_0}\\
    &= \frac{1}{\sqrt{n}} \frac{1}{\sqrt{2^n}}\left[ \sqrt{\frac{m\omega}{\hbar}}x - \sqrt{\frac{\hbar}{m\omega}}\frac{d}{dx} \right]^n \phi_0(x)\\
    \phi_n(x) &= \left[ \frac{1}{2^nn!} \left(\frac{\hbar}{m\omega}\right)^n \right]^{1/2} \left(\frac{m\omega}{\pi \hbar}\right)^{1/4} \left[ \frac{m\omega}{\hbar}x-\frac{d}{dx} \right]^n \text{exp} \left\{-\frac{m\omega}{2\hbar}x^2\right\}
\end{split}
\ee
We see that the general solution is the expotential times a polynomial of degree n and parit (-1)$^n$.
These polynomialsl are well known as the \textbf{Hermite Polynomials}. 
If you plot htese functions (and their probability densities) uyou will see that there are as many poles as excitations (n).
This suggestst the kinetic energy of the particle increasas (increased curcature i.e. larger second derivitive i.e. larger kinetic energy). 
At large n we observe the probability distribution is maximized at the ends of the amplitude.
This is consistent with classical mechanics.
Think of a mass on a spring, at the turning point the velocity must go to zero, therefore the mass spends most of its time at the turning point amplitude. 
In the QM case we have exactlyt he opposite behavior (it is a gaussian maximize at the middle) bbut fo rlarge values of n we recove the classical behavior. 

This result is a general observation of the \textbf{Correspondence Principle} which states that higher energy quantum systems begin to represent classical systems. 

\subsection*{Name}
The eigenstates are all stationary states, nothing is happening.
If we want dynamics we need to look at superpositions. 
In the future we will look at coherence states with the HO to get superpositions.

Neither X or P commute with H, and the eigenstates of H are not eigenstates of X or P. 
We can calculate the expectation values of X and P in a stationary state and verify the uncertainty relationship.
We have already found $X\ket{\phi_n}$ and $P\ket{\phi_n}$ so we can explicitly compute the matrix elements if we want.
But from intuition we already know that
\be
\begin{split}
    \braket{\phi_n|X|\phi_n} &= 0\\
    \braket{\phi_n|P|\phi_n} &= 0
    \end{split}
    \ee
We are in a stationary state, if we have an average value here we cannot be in a stationary state!

We can compute the root-mean-square properties using (for a generic operator A)
%=================================================================================================================%
% Someone should prove the variance proof here if we want
%=================================================================================================================%
\be
\text{var}[A] = \braket{A^2} - \braket{A}^2
\ee
Applying this to X and P we have (remember the average X and P are 0 for the stationary states). 
\be
\begin{split}
    (\Delta X)^2 &= \braket{\phi_n|X^2|\phi_n} - \left(\braket{\phi_n|X|\phi_n}\right)^2 = \braket{\phi_n|X^2|\phi_n}\\
    (\Delta P)^2 &= \braket{\phi_n|P^2|\phi_n} - \left(\braket{\phi_n|P|\phi_n}\right)^2 = \braket{\phi_n|P^2|\phi_n}
\end{split}
\ee

\be
\begin{split}
    X &= \sqrt{\frac{\hbar}{m\omega}}\hat{X}, \qquad \hat{X} = \frac{1}{\sqrt{2}}\left(a^\dg + a\right) \\
    X^2 &= \frac{\hbar}{2m\omega} \left(a^\dg + a\right) \left(a^\dg + a\right)\\
    &= \frac{\hbar}{2m\omega} \left\{(a^\dg) ^2 + a^\dg a + a a^\dg + a^2\right\}
\end{split}
\ee
We need to note that a$^2$ and a$^\dg$ are 0 because $a^2\ket{\phi_n}$ is proportinal to $\ket{\phi_{n-2}}$ and is orthogonal to $\ket{\phi_n}$ and the same with the dagger (they move two excitations). 
We can therefore write (applying the coommutator [a,a$^\dg$])
\be
    X^2 = \frac{\hbar}{2m\omega} \left\{ a^\dg a + a a^\dg\right\} = \frac{\hbar}{2m\omega} \left\{ a^\dg a + 1 + a^\dg a \right\} = \frac{\hbar}{2m\omega} \left\{ 2a^\dg a + 1\right\} = \frac{\hbar}{2m\omega} \left\{ 2N + 1\right\}
\ee
And from here we have evaluated our variance for the HO
\be
(\Delta X)^2 = \braket{\phi_n|X^2|\phi_n} = \left(n + \frac{1}{2}\right) \frac{\hbar}{m\omega}
\ee
%======================================================================================================%
% Someone else can insert the derivation for momentum it follows the same logic
%======================================================================================================%
In the same manner we find
\be
(\Delta P)^2 = \braket{\phi_n|P^2|\phi_n} = \left(n + \frac{1}{2}\right) m\hbar\omega
\ee
Unsurprisingly we see that the variance increases with the quantum number n. 
We can now evaluate teh uncertainty principle
\be
\Delta X \Delta P = \sqrt{(\Delta X^2)(\Delta P^2)} = \left(n+ \frac{1}{2}\right)\hbar
\ee
We see that the lower bound occurs in the ground state (n=0, $\frac{\hbar}{2}$).
The uncertainty principle holds which is epected, X and P do not commute whichh is the root of the uncertainty. 





















\end{document}
